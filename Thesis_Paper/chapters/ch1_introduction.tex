As society increasingly operates in a data-driven world, attention is drawn to the appropriate utilization of data to facilitate general planning, generate predictive analysis, and detect abnormalities. During the outbreak of the Covid-19 pandemic, data has been specifically collected to aid in political planning, damage control, forecasting spread curves, and early warning. However, concerns have arisen with the immense amount of data flowing in every day during the pandemic, particularly with regard to the potential for abuse of metadata collected for the public good. Such concerns center on the leakage of sensitive information, including but not limited to an individual's diagnosis with coronavirus, sexuality, and contact-related activities. Deliberate attackers may use such personal information leakage to perpetrate identity theft. Consequently, it is vital to promote legal and appropriate use of metadata.

To this end, we present a data synthesis and evaluation framework for synthesizing alternative datasets that preserve the data utility of the COVID-19 Trends and Impact Surveys (CTIS) dataset, which tracks social and economic impacts of the pandemic on individuals and households. The synthesis process involves the replacement of original instances with synthetic values using sequential modeling-based and generator networks based data synthesizers. The synthesizers can be more specifically divided into parametric data synthesizers, non-parametric data synthesizers and generative networks with the exponential mechanism (GEM) data synthesizer. We also outline a two-step data synthesis process, in which step 1 involves the synthesis of categorical variables using the above data synthesizers, while step 2 involves the synthesis of the survey weight variable adjusted according to the Facebook user population. The overall synthesis process is divided into four stages: data preprocessing, synthesis with normal variables, synthesis with survey weight, and evaluation of synthetic datasets based on overall and univariate data utility evaluation, inference from fitted linear regression models (analysis-specific utility measurement), and analysis based replication of records. In particular, we employ two subsets of predictors to build linear regression models concerning the installation of contact-tracking apps and diagnoses with the virus. The parameter fit is then assessed by overlapping of confidence intervals.

The paper is organized in a structured manner. In Section 1, we introduce the topic of synthetic data and its importance in preserving privacy while maintaining data usability. Section 2 provides a literature review of data privacy and an overview of the various approaches to data synthesis. In Section 3, we further explore the concept of synthetic data, including methods for obtaining valid statistical inferences and an introduction to selected data synthesis algorithms. Section 4 focuses on the evaluation of synthetic data and risk assessment. We present several metrics for evaluating the utility of synthetic data and methods for assessing the disclosure risk of the synthetic data. In Section 5, we discuss the empirical evaluation of synthetic datasets generated from the CTIS dataset. This section covers the data preprocessing steps, experimental settings, and methodology and workflow explanation. It also presents the results generated from exploratory analysis, including data utility, inference from fitted linear regression models, and replication analysis. Finally, other insights from the application are discussed in this section.




