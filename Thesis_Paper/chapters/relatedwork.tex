In this chapter, general definition of data privacy and the necessity to maintain data privacy are described. 
Furthermore, in order to keep data privacy, this chapter also presents an overview of methods to achieve the 
goal of privacy preserving data analysis and publication, which have been adopted in several domains. 
Here, we want to emphasize the use of synthetic data and one of its most popular applications, 
i.e., differentially private synthetic data, that are able to prevent disclosure in the process of synthetic data generation.

\subsection{Data Privacy}
\label{subsec:dataprivacy}
It is well-acknowledged that we have entered a data-driven world and data are 
often regarded as significant constituents for our society. At the same time, 
an open society can also learn from these data so as to develop feasible 
and practical policy guidelines \citep{evans2021statistically}. Especially during the outbreak of coronavirus 
disease 2019 (COVID-19), more and more increased concerns are raised that it is 
essential for a society to utilize such data, which are widely-spread in the population
and analyzed with regard to various perspectives, to advance sophiscated planning
and develop more concrete social welfare benefits for the citizens. Consequently,
both seen from the public health perspective and the economy perspective, the on-going
COVID-19 global pandemic serves as a rigid reminder that detailed data are urgently 
needed to assist in decision making, damage control scenarios. Regardless of prevalent
consensus reached to leverage more microdata, the inappropriate use of such 
information can cause harm in data confidentiality and privacy as sometimes the attacks
from an intruder may result in the leakage of an individual's sensitive information, e.g., identity, 
address and salary, etc.


On the premise of possible outcomes brought by the misuse of microdata, it is crucial that we encourage
proper and legal use of the collected datasets.

