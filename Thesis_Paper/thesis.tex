\documentclass[12pt]{article}

% \usepackage[showframe]{geometry}
\usepackage{graphicx}
\usepackage{rotating}
\usepackage{booktabs}
\usepackage{ltablex}
\usepackage{ragged2e}
\usepackage{float}
\usepackage{makecell}
\renewcommand{\theadfont}{\normalsize}
\usepackage[]{graphicx}
\usepackage[]{color}
\usepackage{alltt}

\newcommand{\mytitle}{Towards Data Privacy: Evaluating different synthetic data approaches with the data from Covid-19 Trends and Impact Surveys}
\newcommand{\myname}{Yue Xiong}
\newcommand{\mysupervisor}{Dr. Anna-Carolina Haensch}

\usepackage[a4paper, width = 160mm, top = 35mm, bottom = 30mm, 
bindingoffset = 0mm]{geometry}
\usepackage[utf8]{inputenc}
\usepackage{ragged2e}
\usepackage{amsmath}
\usepackage{kantlipsum}
\usepackage{multirow}
\usepackage{stackengine}
\usepackage{lipsum} % just for dummy text- not needed for a longtable
\usepackage{tabularray}
\usepackage{multirow}
% \usepackage[capposition=top]{floatrow}
\usepackage{algorithm,algpseudocode}
\usepackage{caption}
\usepackage{graphicx}
\usepackage{booktabs}
% \usepackage{subfigure}
\usepackage{float}
\usepackage{amssymb}
\usepackage{bbm}
\usepackage{ragged2e}
\usepackage{babel,blindtext}
\usepackage{afterpage}
\captionsetup[table]{position=above}
\usepackage{nccmath}
\usepackage{xcolor}
\usepackage[flushleft]{threeparttable}
\definecolor{mintbg}{rgb}{.63,.79,.95}
\definecolor{lightblue}{RGB}{222,235,247}

\colorlet{lightmintbg}{mintbg!40}
% \usepackage[table]{xcolor}
% \usepackage[capposition=below]{floatrow}
\usepackage{nicematrix}
\usepackage{subcaption}
\usepackage[round, comma]{natbib}
\usepackage{fancyhdr}
\newcommand{\changefont}{%
    \fontsize{6}{11}\selectfont
}
\usepackage{hyperref}
\hypersetup{
  colorlinks = true,
  linkcolor = black,
  urlcolor = black,
  citecolor = black}
\pagestyle{fancy}
\fancyhead{}
\fancyhead[R]{\changefont{\mytitle}}
\fancyfoot{}
\fancyfoot[R]{\thepage}
\setlength{\headheight}{14.5pt}
\setlength{\parindent}{0pt}
\interfootnotelinepenalty = 10000

% ------------------------------------------------------------------------------
% MAIN -------------------------------------------------------------------------
% ------------------------------------------------------------------------------
\IfFileExists{upquote.sty}{\usepackage{upquote}}{}
\begin{document}

% FRONT PAGE -------------------------------------------------------------------
 
\begin{titlepage}
\begin{center}
    
\LARGE
Master's Thesis
    
\vspace{0.5cm}
      
\rule{\textwidth}{1.5pt}
\LARGE
\textbf{\mytitle}
\rule{\textwidth}{1.5pt}
   
\vspace{0.5cm}
      
\large
Department of Statistics \\
Ludwig-Maximilians-Universität München 

\vfill

\Large
\textbf{\myname}

\vfill

\large
Munich, February 27\textsuperscript{th}, 2023
      
\vfill
\includegraphics[width = 0.4\textwidth]{sigillum.png}
\vfill

\normalsize
Submitted in partial fulfillment of the requirements for the degree of M. Sc.
\\

Supervised by \mysupervisor

\end{center}
\end{titlepage}

% CONTENTS ---------------------------------------------------------------------

\pagenumbering{Roman}
\newpage

\begin{abstract}
{\sloppy In the wake of the COVID-19 pandemic, the utilization of disease transmission and contact-related data to predict infection, facilitate quarantine planning, and promote vaccination has been widely acknowledged. However, the sensitive nature of such data demands careful handling to ensure privacy and prevent disclosure of confidential information. In this regard, privacy protection methods offer a solution to extract valuable patterns while safeguarding individuals' privacy. Over the past decade, researchers have actively proposed and developed stronger definitions and methodologies to achieve data privacy while preserving data utility. In the statistics field, synthetic data creation has become one of the most prevalent options for establishing data privacy, and numerous strategies continue to be presented based on this concept. The generation of synthetic data mimics the statistical patterns and characteristics of a confidential data source in a privacy-preserving manner, preserving the original data format and utility in the synthetic versions that can be analyzed safely, without risking privacy violation. Despite its promising potential, this privacy approach has found limited applications in "mimicking" coronavirus related data sources beyond academic research. To investigate the feasibility of private synthetic data generation in real-world scenarios, we explore the potential of this approach based on the COVID-19 Trends and Impact Surveys (CTIS) dataset. We proposed a four-stage data synthesis and utility evaluation framework that provides a unified approach for assessing and comparing various data synthesizers. By examining real-world use cases with this framework, we have incorporated a user-oriented perspective to generate synthetic datasets while examining their performances using the CTIS dataset. Finally, we implemented multiple data synthesis algorithms and evaluate their output synthetic data based on our framework to discuss and compare the quality of private synthetic data generation in terms of data utility and risk assessment.}



% This framework has extended academic evaluation methods by incorporating a user-oriented perspective to provide data synthesis based on different types of variables while examining performance on real-world use cases using the CTIS dataset. 

\end{abstract}

\newpage
\tableofcontents

%%%% if you would want to include material overview
%%%% use one of the following in addition
\clearpage
\addcontentsline{toc}{section}{List of Figures}
\listoffigures \newpage
\addcontentsline{toc}{section}{List of Tables}
\listoftables \newpage
% \newpage
% \listoffigures
% \newpage
% \listoftables
% \newpage
% \listofalgorithms
% \newpage

% CHAPTERS ---------------------------------------------------------------------

\pagenumbering{arabic}
    
\section{Introduction}
\label{chapter1:intro}
As society increasingly operates in a data-driven world, attention is drawn to the appropriate utilization of data to facilitate general planning, generate predictive analysis, and detect abnormalities. During the outbreak of the Covid-19 pandemic, data has been specifically collected to aid in political planning, damage control, forecasting spread curves, and early warning. However, concerns have arisen with the immense amount of data flowing in every day during the pandemic, particularly with regard to the potential for abuse of metadata collected for the public good. Such concerns center on the leakage of sensitive information, including but not limited to an individual's diagnosis with coronavirus, sexuality, and contact-related activities. Deliberate attackers may use such personal information leakage to perpetrate identity theft. Consequently, it is vital to promote legal and appropriate use of metadata.

To this end, we present a data synthesis and evaluation framework for synthesizing alternative datasets that preserve the data utility of the COVID-19 Trends and Impact Surveys (CTIS) dataset, which tracks social and economic impacts of the pandemic on individuals and households. The synthesis process involves the replacement of original instances with synthetic values using sequential modeling-based and generator networks based data synthesizers. The synthesizers can be more specifically divided into parametric data synthesizers, non-parametric data synthesizers and generative networks with the exponential mechanism (GEM) data synthesizer. We also outline a two-step data synthesis process, in which step 1 involves the synthesis of categorical variables using the above data synthesizers, while step 2 involves the synthesis of the survey weight variable adjusted according to the Facebook user population. The overall synthesis process is divided into four stages: data preprocessing, synthesis with normal variables, synthesis with survey weight, and evaluation of synthetic datasets based on overall and univariate data utility evaluation, inference from fitted linear regression models (analysis-specific utility measurement), and analysis based replication of records. In particular, we employ two subsets of predictors to build linear regression models concerning the installation of contact-tracking apps and diagnoses with the virus. The parameter fit is then assessed by overlapping of confidence intervals.

The paper is organized in a structured manner. In Section 1, we introduce the topic of synthetic data and its importance in preserving privacy while maintaining data usability. Section 2 provides a literature review of data privacy and an overview of the various approaches to data synthesis. In Section 3, we further explore the concept of synthetic data, including methods for obtaining valid statistical inferences and an introduction to selected data synthesis algorithms. Section 4 focuses on the evaluation of synthetic data and risk assessment. We present several metrics for evaluating the utility of synthetic data and methods for assessing the disclosure risk of the synthetic data. In Section 5, we discuss the empirical evaluation of synthetic datasets generated from the CTIS dataset. This section covers the data preprocessing steps, experimental settings, and methodology and workflow explanation. It also presents the results generated from exploratory analysis, including data utility, inference from fitted linear regression models, and replication analysis. Finally, other insights from the application are discussed in this section.





\newpage

\section{Related Work}
\label{chapter2:relatedwork}
\input{chapters/ch2_relatedwork}
\newpage

\section{Synthetic Data}
\label{chapter3:syn}
Following an overview of the evolvement of data privacy preservation in synthetic data, and a discussion of computer science-based and statistics-based approaches, this chapter now examines synthetic data from a formal mathematical perspective. Therefore, firstly, based on the sequential modelling strategy introduced in section \ref{chapter2:relatedwork}, the mathematical formulation of synthetic data with sequential modelling is given with intuitive understanding. Then, implications on how to obtain valid inference of synthetic data is presented, where the combining rules is mainly discussed. Subsequently, some detailed data synthesis algorithms are depicted with a focus on parametric and non-parametric methods. At the end of chapter \ref{chapter3:syn}, concluding remarks are provided based on the comparison between parametric and non-parametric synthetic data approaches.

\subsection{Synthetic Data with Sequential Modelling}
\label{subsec:syntheticData}
As mentioned in chapter \ref{chapter1:intro}, tremendous amounts of data are collected about individuals by a variety of parties, especially during the outbreak of the COVID-19 Pandemic, to advance decision making and damage control scenarios. Even though this data collection allows to analyze data to bring lots of benefits to the society, the major concern about privacy preservation is growing exponentially. The idea of Synthetic Data has managed to offer an intuitive way to resolve this issue by imputing these sensitive data entries with synthesized data instances where the level of privacy protection differentiates the research branches into fully synthetic data and partially synthetic data.

Among all statistical approaches given in section \ref{subsubsec:statsapproach}, a brief introduction to the sequential modelling technique is given, for which every variable is sequentially generated using models predicated on variables that have previously been generated or have remained constant. The sequential modelling technique, also known as Fully conditional specification (FCS), replaces the challenge of drawing from a $k-$variate distribution with a simpler task of drawing from univariate distributions. Each variable in the dataset is considered individually and imputed using a regression model appropriate for that specific variable. More specifically, continuous variables can be imputed through a normal model, binary variables can be imputed using a logit model, and multivariate variables can be imputed with multiclass classification model such as CART. With this approach, the conditional probability $P(\theta|Y_{obs})$ can be specified directly, eliminating the need for iterations. This is due to the fact that drawing from potentially complex multivariate distributions is not required in this case. For instance, consider the case where values for a continuous variable $Y$ are to be imputed. The conditional distribution $Y|X$ can be modeled as $N(\mu ,\sigma^2)$, where $X$ represents all variables that serve as explanatory variables for the imputation process. The two-step imputation procedure can then be executed as follows: Let $n$ be the number of observations in the dataset, $k$ be the number of regressor variables included in the regression, $\sigma^2$ be the variance estimate obtained from ordinary least squares regression, and $\hat{\beta}$ be the beta-coefficient estimate acquired from the same regression. Moreover, in the context of missing data, it is assumed that plausible starting values for the missing portion of $Y$ have been filled in or generated in prior imputation rounds. These starting values can be collected, for instance, through the use of predicted values from a linear regression of $Y$ on $X$. The imputed values for $Y_{new}$ can be produced through the application of the following algorithm:
\begin{algorithm}[H]
\caption{Data imputation for $Y_{new}$, see \cite{drechsler2011synthetic}}
\begin{algorithmic}[1]
\State Step 1: Draw new values for $\theta=(\sigma^2, \beta)$ from $P(\theta|Y)$; i.e.,
\State \hskip1.0em draw $\sigma^2|X \sim (Y-X\hat{\beta})^{'}(Y-X\hat{\beta})\chi_{n-k}^{-2},$
\State \hskip1.0em draw $\beta|\sigma^2, X \sim N(\hat{\beta},(X^{\prime}X)^{-1}\sigma^2).$
\State Step 2: Draw new values for $Y_{new}$ from $P(Y_{new}|Y,\theta)$; i.e.,
\State \hskip1.0em draw $Y_{new}|B, \sigma^2,X \sim N(X\beta,\sigma^2).$
\end{algorithmic}
\end{algorithm}

It is important to mention that new values for the parameters are being drawn directly from the posterior distributions of the observed data. This eliminates the need for Markov Chain Monte Carlo methods to derive new values from the complete-data posterior distribution of the parameters. However, in scenarios where there are multiple variables with missing data, values for Ynew are generated by sampling from $P(Y_{new} | \beta, \sigma^2, X)$, where $X$ may contain imputed values from a previous iteration. These values must be updated based on the new information obtained from the recently imputed variable $Y$. As a result, a Gibbs sampler must be applied iteratively to every variable in the dataset in order to sample from the fully conditional distribution. Assuming the corresponding joint distribution exists, this iterative procedure effectively converges to draws from the joint distribution.


With a more elaborate indication, for multivariate $Y$, consider $Y_{j}|Y_{-j}$ to represent the conditional distribution of $Y_{j}$ given the columns of $Y$ excluding $Y_{j}$, and $\theta_{j}$ as the parameter defining the distribution of $Y_{j}|Y_{-j}$. Thinking of a scenario where the multivariate $Y$ is comprised of $p$ columns, and each column, $Y_j$, is univariate. The $t$-th iteration of the specified method involves successive draws, executed in the following manner, as proved by \citet{drechsler2011synthetic}:

\begin{align}
\label{eqn:eqlabel}
\begin{split}
    \theta_{1}^{(t)} &\sim P(\theta_{1}|Y_{1}^{(t-1)},...,Y_{p}^{(t-1)})\\
    Y_{1}^{(t)} &\sim P(Y_{1}^{new}|Y_{2}^{(t-1)},...,Y_{p}^{(t-1)}, \theta_{1}^{(t)})\\
    &\cdots \\
    \theta_{p}^{(t)} &\sim P(\theta_{p}|Y_{p}^{(t-1)},Y_{1}^{(t)},Y_{2}^{(t)},...,Y_{p-1}^{(t)})\\
    Y_{p}^{(t)} &\sim P(Y_{p}^{new}|Y_{1}^{(t)},...,Y_{p-1}^{(t)}, \theta_{p}^{(t)}).
\end{split}
\end{align}
Due to the fact that these data imputations are synthesized in a sequential manner, this technique is also termed as sequential regression multivariate imputation (SRMI; refer to \cite{raghu2001information}). It is worth mentioning that the attainment of the desired joint distribution of $(Y_{new}|Y_{obs})$ by the sampler is contingent upon the actual existence of such a joint distribution. In practical applications, verification of the existence of the joint distribution of $(Y_{new}|Y_{obs})$ is often infeasible. However, this poses a challenge as it is always possible to sample from the conditional distributions, potentially obscuring the fact that the Gibbs sampler may not have actually converged.


An effective approach to identifying issues with the iterative imputation procedure is to record the mean of each imputed variable during each iteration of the Gibbs sampler. Plotting the means of the imputed variables over the course of the iterations allows for the determination of whether there is the expected random fluctuation or if there is a discernible pattern, which would indicate problems with the model. Furthermore, it is important to note that the absence of a noticeable trend over the iterations does not necessarily guarantee convergence, as the monitored estimates may remain stable for numerous iterations before diverging to infinity. Despite this, the method of tracking the means of the imputed variables is a simple approach to detect faulty imputation models. To further monitor convergence, one can compute the variance of a specified estimate of interest $\Psi$, such as the mean and standard deviation of each variable, both within and across multiple imputation chains if different chains are utilized to generate the multiple imputations. Let $\Psi_{ij}$ represent the estimate obtained at iteration $i$ (ranging from 1 to $T$) in chain $j$ (ranging from 1 to $m$). The variance between sequences, $B$, and the average variance within sequences, $W$, can be computed as follows:
\begin{align*}
  B = \frac{T}{m-1}\sum_{j=1}^{m}(\Psi_{.j}-\Psi_{..})^{2},\;&\text{where}\; \Psi_{.j}=\frac{1}{T}\sum_{i=1}^{T}\Psi_{ij}, \Psi_{..}=\frac{1}{m}\sum _{j=1}^{m}\Psi_{.j},\\
  W = \frac{1}{m}\sum_{j=1}^{m}s_{j}^{2},\;&\text{where}\; s_{j}^{2} = \frac{1}{T-1}\sum_{i=1}^{T}(\Psi_{ij}-\Psi_{.j})^2.
\end{align*}

The convergence of Gibbs sampler is assumed, according to \citet{gelman2004parameterization}, when 
\begin{align}
\label{eqn:converge}
    \hat{R}=\sqrt{\frac{(1-1/T)W+B/T}{W}}<1.1
\end{align}

In addition, it should be mentioned that the iteration process between imputations is not always required. If the data can be rearranged in a way that ensures $Y_j$ is completely observed whenever $Y_{j+1}$ is observed, a modified sequential regression algorithm can be employed that eliminates the need for iteration between imputations. In this scenario, $X$ represents all the variables in the dataset that have full observations, and $Y_1,...,Y_p$ represents the variables with missing values, ordered based on the extent of missingness. 
% \begin{table}[H]
%     \centering
%     \caption{A non-floating table with \texttt{H} option}
%     \begin{tabular}{cc}
%         \toprule
%          Ducks & Lions \\
%          \midrule
%          1 & 2 \\
%          \bottomrule
%     \end{tabular}
% \end{table}
\begin{figure}[H]
    \centering
    \includegraphics[width=.8\linewidth]{graphics/Fig-1-missingness-pattern.png}    
    \caption{Two missing data patterns.}
    \label{fig:missingpatterns}
\end{figure}

% Just to show the also the lists works:
% \listoftables
% \listoffigures
As shown in Figure \ref{fig:missingpatterns}, the illustration presents two distinctive patterns of missing data. The pattern on the left is characterized as having a monotonic missingness pattern, as the number of missing observations increases in a monotonic fashion from $Y_1$ to $Y_p$. The pattern on the right, however, is non-monotonic, as there are instances where values are available for $Y_{j+1}$ but not for $Y_j$.

Given that the joint probability of a dataset can be always expressed as the product of conditional probabilities:
\begin{align}
\label{equ:jointdis}
  P(Y_1,...,Y_p|X)=P(Y_1|X)\times P(Y_2|Y_1,X)\times ... \times P(Y_p|Y_1,...,Y_{p-1},X).
\end{align}

In the presence of a monotone missingness pattern, as $Y_j$ is observed, so will $Y_1,\ldots,Y_{j-1}$, meaning that the conditional distributions are unchanged through the imputation of the missing values in $Y_1,\ldots,Y_{j-1}$. Thus, the parameters do not have to be updated with each iteration. Each draw will be made directly from the posterior distribution, eliminating the need for convergence.

However, in most real-world data collections, the missingness pattern is not monotone unless it was intentionally designed that way, for example, by conducting a follow-up study only with a subset of original survey participants. On the other hand, when generating synthetic datasets through multiple imputation, it is common to replace the same number of records for each sensitive variable, meaning that the decision of whether a value is missing is based on the combined attributes of the record, not just the variable. This implies that when generating synthetic datasets, the simplified algorithm can often be used, reducing the time needed to generate the datasets and eliminating the need to monitor convergence.

As noted in section \ref{subsubsec:statsapproach}, besides sequential modelling, we also have the joint modelling technique, which directly specifies or assumes the joint distribution of the original dataset and later utilizes this distribution to synthesize new data instances. Generally, data in practice will not follow a standard multivariate distribution, especially if it includes a mixture of numerical and categorical variables. The sequential modelling technique provides a flexible tool to accommodate for bounds, interactions, skip patterns, or restrictions between variables. In contrast, it can be challenging to handle these restrictions with joint modeling. Often, imputation is centralized in the methodological department of a statistical agency, and imputation experts will perform imputation for all the surveys conducted by the agency. If the imputed datasets do not adhere to simple restrictions such as non-negativity or logical constraints, they will not be accepted by subject matter analysts from other departments. Hence, maintaining these restrictions is crucial in the imputation task, making the sequential modelling technique a preferred method for most applications of multiple imputation.

In summary, joint modeling is preferred for imputation tasks involving a limited number of variables with no restrictions and well-approximated joint distributions with a standard multivariate distribution. However, for complex imputation tasks, data synthesis with sequential modelling is necessary to maintain the constraints inherent in the data. It is important to monitor the convergence of the Gibbs sampler in such cases.




% Regardless of which functions $f_i$ are utilized for imputing the original data instances, it is also important to ensure the validity of inferences from the altered data. As for the next section, we will discuss how to obtain valid inferences for the Multiple Imputation inspired synthesizing approaches \citep{rubin1993statistical} based on a broader application scenario with a general imputation methodology. 



\subsection{Obtaining Valid Statistical Inferences}
\label{subsec:inference}
As stated in section \ref{subsec:datasynthesis}, the inception of Rubin's proposal for synthesizing data was driven by his previous research on multiple imputation to handle nonresponse. Given the close association with those related concepts, it is widely accepted to utilize straightforward combination techniques derived from previous research on multiple imputation (Rubin's combining rules) when seeking to gain valid point and variance estimates through the use of synthesized data. However, the methodologies of synthetic data generation deviate from the framework established by Rubin in two significant aspects, depending on the level of protection required. In the context of full synthesis, as initially proposed by \citet{rubin1993statistical}, fully synthetic data is only produced for a randomly selected subset of the population, necessitating careful consideration of the additional sampling step. In contrast, during partial synthesis, the synthesis models are estimated using the complete data, not solely the available subset, as is typically done in the context of nonresponse. As a result of these deviations, the combining procedures must also be adapted accordingly.

To analyze multiply imputed synthetic datasets, consider a scenario in which an analyst is interested in an unknown scalar parameter $Q$. This parameter could be a mean of a variable, a correlation coefficient between two variables, or a regression coefficient in a linear regression. Assuming there are no missing data in the observed dataset, inferences for $Q$ are usually based on a point estimate ($q$), an estimate of the variance of $q$ ($u$), and a normal or Student's $t$ reference distribution. For analysis of the synthetic datasets, let $q^{(i)}$ and $u^{(i)}$ be the point and variance estimates for each of the m synthetic datasets, where $i = 1, ..., m$. To make inferences for scalar $Q$, the following quantities are necessary:
\begin{align}
\label{equ:qm}
\bar{q}_{m} &= \sum_{i=1}^{m}q^{(i)}/m,\\
\label{equ:bm}
b_m &= \sum_{i=1}^{m}(q^{(i)}-\bar{q}_{m})^2/(m-1),\\
\label{equ:um}
\bar{u}_{m} &= \sum_{i=1}^{m}u^{(i)}/m.
\end{align}
which are provided by \citet{drechsler2011synthetic}. In the subsequent sections, we are going to introduce the combining rules for fully synthetic data and partially synthetic data, respectively, based on the mentioned quantities \eqref{equ:qm}, \eqref{equ:bm}, and \eqref{equ:um}.

\subsubsection{Combining Rules for Full Data Synthesis}
\label{subsubsec:fullSyn}
An unbiased point estimate of $Q$ can be obtained by employing the $\bar{q}_m$ estimator, and its variance can be estimated through the use of the equation provided as follows:
\begin{align}
    \label{equ:tf-fully}
    T_{f}=(1+m^{-1})b_m-\bar{u}_m.
\end{align}

When the sample size $n$ is large, the scalar parameter $Q$ can be estimated using a $t-$distribution with degrees of freedom calculated as $v_p = (m - 1)(1 + \bar{u}_m/((1+m^{-1})b_m))^2$. However, the variance estimate $T_f$ can sometimes be negative, so a modified variance estimator is suggested by \citet{reiter2002satisfying} that always produces a positive value. The modified variance estimator, $T_{f}^{*} = \text{max}(0,T_{f}) + \delta(\frac{n_{syn}}{n}\bar{u}_m)$, adds a constant value ($\delta$) to the original estimate to ensure positivity, where $\delta$ is equal to 1 if $T_f$ is negative and 0 otherwise. $n_{syn}$ represents the number of observations in the synthetic datasets used for analysis. Note that this modified variance estimator has a tendency to overestimate the true variance of $T_f$.



\subsubsection{Combining Rules for Partial Data Synthesis}
\label{subsubsec:partialSyn}
In like manner, when it comes to the combining rules for partially synthetic data, one can estimate $Q$ by utilizing  $\bar{q}_m$. The variance of $\bar{q}_m$ for this type of synthetic data can then be computed utilizing an appropriate method:
\begin{align}
    \label{equ:tf-part}
    T_{p}=b_m/m + \bar{u}_m.
\end{align}

Especially for large sample sizes $n$, the estimation of the scalar parameter $Q$ can be based on $t-$distributions with degrees of freedom $v_p$, which is calculated as $v_p = (m - 1)(1 + \bar{u}_m/(b_m/m))^2$. Additionally, it is important to note that the variance estimate $T_p$ cannot be negative, so there is no need for adjustments in case of partially synthetic datasets.

\subsubsection{An Alternative Variance Estimate for Full Data Synthesis}
\label{subsubsec:alt-var-est}
In fully synthetic data generation, the practice followed by most researchers deviates from the original protocol proposed by \citet{rubin1993discussion}. Rubin's assumption was that in addition to the survey variables $Y$, additional variables $X$ representing the design variables from the sampling frame would be available in the full population. Based on this assumption, the generation of fully synthetic data for $Y$ would involve the fitting of a conditional model for $f(Y | X)$ using the survey data, and using this model to generate synthetic values for $Y$ with a new sample of design variables $X^{new}$ drawn from $f(Y | X^{new})$. Note that only the synthetic $Y$ values would be afterwards made available to the public.

Nevertheless, most researchers in practice only utilize the information in $Y$ for synthetic data generation. This method of generating fully synthetic data can be considered an extreme form of partial data synthesis with an empty set of unsynthesized records. As noted by \citet{drechsler2011improved}, the combining rules for partial synthesis remain valid in this context. Building on these concepts, \citet{raab2016practical} has introduced an alternative estimator for the variance to be used in this scenario:
\begin{align*}
    T_{s}=\left ( \frac{n_{syn}}{n_{org}}+\frac{1}{m} \right )\bar{u}_m,
\end{align*}
in which $n_{org}$ indicates the number of instances in the original dataset and $n_{syn}$ refers to the number of synthesized records. Based on the extension provided by \citet{raab2016practical}, it is important to mention that the proposed variance estimator, $T_s$, does not rely on the estimate of the between imputation variance, $b_m$. Due to this, it has actually offered several benefits compared to the variance estimator, $T_f$, discussed previously for fully synthetic data. Firstly, the estimator $T_f$ can never be negative. Secondly, it demonstrates less variability than $T_f$, as $b_m$ is only an estimate of the true variability between datasets and its estimation based on a limited number of synthetic datasets results in high uncertainty. Thirdly, a valid variance estimate can be obtained from a single synthetic dataset. This is particularly relevant because previous research has indicated that the risk of disclosure increases with the number of synthetic datasets released \citep{drechsler2009disclosure,reiter2010releasing}. However, releasing only a single synthetic dataset results in increased uncertainty. Assuming that $n_{syn} = n_{org}$, the variance can be reduced by 25\% when two datasets are released instead of one. However, these accuracy gains are rapidly diminishing with increasing m, and the relative reduction in variance is limited to 0.5 as m approaches infinity. For further discussion of the advantages and disadvantages of different synthesis strategies and appropriate variance estimator selection based on the scenario, please refer to \citet{drechsler2018some}.




\subsection{Introduction to Selected Data Synthesis Algorithms}
\label{subsec:detailsynmethods}
In previous section \ref{subsec:inference}, we have introduced the combining rules for fully and partially synthetic data to obtain valid statistical inference. Now, we begin to illustrate on specific synthesizing algorithms that can be employed to carry out data synthesis tasks. Before giving detailed explanations on different algorithms, it is worth noting that we can categorize them into broader groups where the splitting rule lies mostly in the corresponding synthesizing mechanism. More specifically, the categorization of data synthesis algorithms can be split into the first two groups: parametric data synthesis algorithms, non-parametric data synthesis algorithms. Since this paper is also in cooperation with \citet{liu2021iterative} to generate synthetic data using generative networks with the exponential mechanism (GEM), we will talk about the data synthesis algorithm based on GEM in the third group.

\paragraph{Parametric data synthesis algorithms}rely on the assumption of a specific probability distribution to model the data. This type of synthesis is suitable when the underlying distribution of the data can be accurately estimated and modeled through parametric methods. In general, the benefit of this method is that the data synthesis process is efficient and the synthesized data can be generated quickly.

\paragraph{Non-parametric data synthesis algorithms}do not rely on any assumptions about the underlying distribution of the original data, on the contrary. Instead, they turn to the actual data to generate synthetic data by using estimators based on non-parametric methods, such as CART. This type of synthesis is often applied when the underlying distribution of the data is complex and cannot be accurately estimated through parametric methods. Accordingly, the advantage of this algorithm is that the structure of this type of synthesizer is more flexible, and can additionally offer the inclusion of variables' interactions to study the relationship among the data variables.

\paragraph{GEM-based data synthesis}refers to using generative neural networks with the exponential mechanism, the authors have proved that it circumvents computational bottlenecks in algorithms such as MWEM \citep{hardt2010multiplicative}, also known as a multiplicative weights mechanism for privacy-preserving data analysis by optimizing over generative models parameterized by neural networks, which capture a rich family of distributions while enabling fast gradient-based optimization to provide private synthetic data generation for query release.

\subsubsection{Parametric Data Synthesis Algorithms}
\label{subsubsec:para}
Parametric data synthesis algorithms are a class of data synthesis techniques that rely on the assumption of a specific distributional form for the variables in the target population. In this category, we are going to introduce detailed algorithms for normal linear regression and normal linear regression preserving the marginal distribution.

\paragraph{Normal linear regression}
Normal linear regression \citep{su2012linear} is one such parametric data synthesis algorithm that is widely used in the literature. The algorithm is based on the premise that the target population's variables follow a multivariate normal distribution and that the relationships between the variables can be modeled using linear regression.

Let $Y$ be a $p-$dimensional response variable and $X$ be a $q-$dimensional design matrix of covariates. The model can be represented as:
\begin{align}
    \label{equ:lm-org}
    Y=X\beta+\xi,
\end{align}
where $\beta$ is the $p-$dimensional vector of coefficients and $\xi$ is the $p-$dimensional vector of bias that are normally distributed with mean $0$ and covariance matrix $\sigma^2I$. Here, $I$ represents the $p-$dimensional identity matrix. Accordingly, the covariance matrix can be estimated from the sample data, and the coefficients can be estimated using ordinary least squares regression. As a matter of fact, this procedure generates a synthetic dataset that has the same statistical properties as the original dataset, as specified by the normal linear regression model.

In the scenario of using linear regression as data synthesizer, given a new set of design variables $X_{new}$, the synthetic values for $Y_{new}$ can be generated as follows:
\begin{align}
    \label{equ:lm-syn}
    Y_{new}=X_{new}\hat{\beta}+\xi_{new},
\end{align}
with $\hat{\beta}$ indicating the estimate of the coefficients obtained from the sample data and $\xi_{new}$ is a $p-$dimensional vector of errors that is drawn from a multivariate normal distribution with mean $0$ and covariance matrix $\sigma_{new}^2I_{new}$. Note that $I_{new}$ represents the identity matrix used for synthesis.

\paragraph{Normal linear regression preserving the marginal distribution}
Similar to normal linear regression, the algorithm of normal linear regression preserving the marginal distribution is based on the linear regression technique but maintains the marginal distribution of the original dataset. In detailed manner, the algorithm can be summarized as follows:
\begin{itemize}
    \item Step 1 (Data Preparation): Obtain the original survey data $Y_{org}$ and its corresponding design variables $X_{org}$,
    \item Step 2 (Fit the Regression Model): Establish a linear regression model between $Y_{org}$ and $X_{org}$, denoted as $Y_{org} = \beta_{org}X_{org} + \xi_{org}$, where $\beta_{org}$ is the regression coefficient and $\xi_{org}$ is the residual error. The regression coefficients can be estimated using the maximum likelihood method (MLE),
    \item Step 3 (Determine the Synthetic $X_{syn}$ values): Draw a random sample of synthetic $X_{syn}$ values from the marginal distribution of the design variables $X_{org}$,
    \item Step 4 (Generate Synthetic $Y_{syn}$ values): Use the regression coefficients obtained from Step 2 and the synthetic $X_{syn}$ values generated from Step 3, generate synthetic $Y_{syn}$ values by computing $Y_{syn} = \hat{\beta}_{syn}X_{syn} + \xi_{syn}$, where $\xi_{syn}$ is drawn from the residual error distribution estimated in Step 2.
\end{itemize}
Also note that the regression coefficients $\hat{\beta}_{syn}$ can be estimated by maximizing the likelihood function $L(\beta)$ given by:
\begin{align}
    \label{equ:mle}
    L(\beta)=\prod_{i=1}^{n}(\frac{1}{\sqrt{2\pi \sigma^2}}\text{exp}(-\frac{(Y_{i}-\beta X_{i})^2}{2\sigma^2})),
\end{align}
where $n$ is the sample size, $\sigma^2$ is the residual error variance, and $Y_i$ and $X_i$ are the observed values of $Y_{org}$ and $X_{org}$, respectively. Specifically, the residual error distribution can be estimated as $N(0, \sigma^2)$. By obtaining the marginal distribution of observed $X$, the synthetic $X_{syn}$ can be drawn from this, and the corresponding $Y_{syn}$ of synthetic values can be generated using $Y_{syn}=\hat{\beta}_{syn}X_{syn}+\xi_{syn}$, with $\xi_{syn}$ drawn from the residual error distribution.

In conclusion, both the normal linear regression data synthesizer and the normal linear regression data synthesizer preserving the marginal distribution are parametric data synthesis algorithms that can be utilized to generate synthetic data. The normal linear regression synthesizer generates synthetic data by fitting a regression model using the original data and using the model to predict the values of the synthetic data. On the other hand, the normal linear regression synthesizer preserving the marginal distribution generates synthetic data by first estimating the marginal distribution of the original data and then fitting a regression model using the estimated marginal distribution as a constraint.

The choice of which algorithm to use depends on the specific requirements of the data synthesis task. If preserving the marginal distribution of the original data is important, then the normal linear regression synthesizer preserving the marginal distribution may be the more appropriate choice. However, if the focus is on the relationship between the variables in the data and not the marginal distribution, then the normal linear regression synthesizer may be the better option.


\subsubsection{Non-parametric Data Synthesis Algorithms}
\label{subsubsec:non-para}
In contrast to parametric data synthesizers, non-parametric data synthesizers do not rely on a priori assumptions about the underlying population distribution. Rather, they use flexible, data-driven models to generate synthetic data. Non-parametric data synthesizers offer several advantages, including their ability to adapt to complex, non-linear relationships between variables and their insensitivity to outliers and other anomalies in the data. They are also capable of modeling complex relationships between variables that are difficult to capture with parametric methods. However, non-parametric methods also have their limitations, such as the need for large amounts of data to avoid overfitting and the potential for slower computation times compared to parametric methods. Despite these limitations, non-parametric data synthesizers are an important tool for producing synthetic data and are widely used in applications where the underlying population distribution is unknown or difficult to model using parametric methods.

In the following section, we will delve into the intricacies of non-parametric data synthesis algorithms. In particular, we will examine the methods of polytomous logistic regression, classification and regression trees, random forest, and random forest-based bagging algorithms. These algorithms provide valuable alternatives to parametric data synthesis and have been widely used in various applications due to their flexibility and ability to handle complex relationships between variables. Furthermore, we will study their advantages, limitations, and implementation techniques to provide a comprehensive understanding of non-parametric data synthesis.

\paragraph{Polytomous logistic regression} The polytomous logistic regression (PLR) method is a popular approach in non-parametric data synthesis. This method is used to model the relationship between a categorical response variable and a set of predictor variables. Unlike the traditional logistic regression, which models binary responses, PLR is used to model multi-class response variables with more than two categories.

Given a dataset $D_{org}$ with $n_{org}$ observations and $p$ predictor variables, the goal of PLR is to generate a synthetic dataset $D_{syn}$ with the same statistical properties as the original dataset. To do this, PLR models the conditional probability of each category of the response variable given the predictor variables, using the following formula:
\begin{align}
    \label{equ:multilogit}
    P(y=k|x_1,x_2,\dots,x_p)=\frac{\exp(\beta_{0k}+\beta_{1k}x_1+\beta_{2k}x_2+\dots + \beta_{pk}x_p)}{\sum_{j=1}^K\exp(\beta_{0j}+\beta_{1j}x_1+\beta_{2j}x_2+\dots + \beta_{pj}x_p)},
\end{align}
where $y$ is the response variable, $x_i$ are the predictor variables, $\beta_{ik}$ are the coefficients of the logistic regression model, and $K$ is the number of categories of the response variable.

To generate the synthetic data, the PLR model is first trained using the original data. Then, the model is used to generate synthetic values for each observation in the synthetic dataset. For each observation, the values of the predictor variables are kept constant, and the value of the response variable is sampled from the conditional probability distribution given by the PLR model. The process of fitting the logistic regression models and sampling new response values can be repeated to generate a desired number of synthetic records. Also, it is worth mentioning that the synthetic records generated using PLR will have the same marginal distribution as the original data, but will have a different joint distribution.


\paragraph{Classification and regression trees}
Having discussed the PLR data synthesizer, we now turn our attention to the Classification and Regression Trees (CART) data synthesizer. CART is a popular non-parametric machine learning method that has been widely used in various data synthesis tasks. In the next section, we will provide a comprehensive explanation of how CART can be employed to generate synthetic records and the formulas involved in this process.

The key idea behind CART is to recursively partition the feature space into smaller and smaller regions, such that the target variable is relatively homogeneous within each region. This partitioning process is repeated until a stopping criterion is met, which can be based on the size of the partition, the amount of improvement in the target variable's homogeneity, or the complexity of the tree model. The final tree structure consists of internal nodes, which are used to make decisions about which partition to follow, and terminal nodes, which correspond to the synthetic records.

Let us denote the original dataset as $D = { (x_1,y_1),(x_2,y_2),\dots,(x_n,y_n)}$, where $x_i$ represents the feature vector and $y_i$ represents the target variable, $i=1,...,n$. The goal of CART is to synthesize a new data set $D' = { (x'_1,y'_1),(x'_2,y'_2),\dots,(x'_m,y'_m)}$ that has the same statistical properties as $D$. To generate synthetic records using CART, we first fit the tree model to the original data set $D$. For each terminal node in the tree model, we can obtain the distribution of the target variable by calculating the mean and standard deviation of the target variable in that node. We can then sample synthetic records from this distribution to obtain $(x'_i,y'_i)$. The mean and standard deviation of the target variable in terminal node $t$ can then be formally denoted as $\mu_t$ and $\sigma_t$, respectively. Subsequently, the synthetic records in terminal node $t$ can be generated using the following formula:
\begin{align}
    \label{equ:cart}
    y'i = \sum_{j=1}^{K} w_{ij} y_{j},
\end{align}
where $K$ is the number of terminal nodes in the tree, $y_j$ is the response value in the $j^{th}$ terminal node, and $w_{ij}$ is the weight assigned to the $j^{th}$ terminal node for the $i^{th}$ synthetic record, indicating the probability that the $i^{th}$ synthetic record belongs to the $j^{th}$ terminal node. Note that an illustration with a simple workflow of the CART data synthesizer can be found in Figure \ref{fig:cartsyn}.
\begin{figure}[H]
    \centering
    \includegraphics[width=.75\linewidth]{graphics/Fig-2-cart-synthesizer.png}    
    \caption{A simple workflow of the CART data synthesizer.}
    \label{fig:cartsyn}
\end{figure}

It is worth noting that CART is a flexible algorithm and can handle both categorical and numerical variables, as well as handle non-linear relationships between the features and target variables. However, the tree structure can become complex and difficult to interpret, especially when the number of features is large or the relationship between the features and target variables is complicated.

\paragraph{Random forest}
In contrast to CART, random forest \citep{rigatti2017random} is an ensemble learning method that generates multiple decision trees and combines their results to produce a final prediction. It is a powerful machine learning algorithm that has been widely used in various data synthesis tasks. In the next section, we will provide an overview of how random forest can be employed to generate synthetic records and the key concepts involved in this process.

Let us denote the original dataset as $D = { (x_1,y_1),(x_2,y_2),\dots,(x_n,y_n)}$, where $x_i$ represents the feature vector and $y_i$ represents the target variable, $i=1,...,n$. The goal of the random forest is to synthesize a new dataset $D' = { (x'_1,y'_1),(x'_2,y'_2),\dots,(x'_m,y'_m)}$ that has the same statistical properties as $D$.

To generate synthetic records using random forest, we first fit the random forest model to the original data set $D$. The random forest model will then generate multiple decision trees from bootstrapped samples of $D$ and random subsets of the features. For each record in the original dataset, the random forest model will generate a synthetic record by combining the prediction from all the trees in the forest, where the combination lies in a pre-chosen criteria such as majority voting or averaging.

Assuming averaging is used for the final result combination, we now denote the prediction of the $i^{th}$ tree in the forest for the $j^{th}$ record in the original dataset as $y_{ij}$. The synthetic record for the $j^{th}$ record can then be obtained using the following formula:
\begin{align}
\label{equ:rf}
y'j = \frac{1}{B} \sum_{i=1}^{B} y_{ij},
\end{align}
where $B$ is the number of decision trees in the random forest.

In general, random forest is a powerful algorithm that can handle both categorical and numerical variables and can also handle non-linear relationships between the features and target variables. By combining the predictions from multiple decision trees, it provides a robust way to generate synthetic records that are representative of the original dataset.



\paragraph{Bagging}
Bagging (Bootstrapped Aggregation) is a simple ensemble learning technique that involves fitting multiple models on different samples of the data and combining their predictions. Based on this understanding, we can also think of random forest as a type of bagging, which generates multiple decision trees from bootstrapped samples of the original data and random subsets of features, and combines the results of these trees to make a final prediction. With a focus on random forest based bagging data synthesis, we now would like to present a detailed illustration of how this data synthesizer works to output synthetic records.

Let us denote the original dataset as $D = { (x_1,y_1),(x_2,y_2),\dots,(x_n,y_n)}$, where $x_i$ represents the feature vector and $y_i$ represents the target variable, $i=1,...,n$. The goal of the random forest based bagging data synthesizer is to synthesize a new dataset $D' = { (x'_1,y'_1),(x'_2,y'_2),\dots,(x'_m,y'_m)}$ preserving the same properties as $D$.

Specifically, we firstly fit the random forest model to the original data set $D$. The random forest model generates multiple decision trees from bootstrapped samples of $D$ and random subsets of the features. Seeing each random forest model as a single training unit, we then utilize multiple random forest models to fit for different random sets of the original dataset. Finally, the prediction is summarized with a combination rule. By sampling records from the output prediction's distribution, we will get the corresponding synthetic dataset. Note that using random forest as a single fitting unit can further reduce the correlation between trees and therefore increase diversity in the ensemble.

To conclude, both parametric and non-parametric data synthesizers have their own pros and cons in terms of model complexity, efficiency and training cost, ability to study interaction or complex relationships between variables, and ability to overcome collinearity issues.

Parametric data synthesizers, such as regression-based methods, have the advantage of being simple and efficient to implement. They have a low model complexity and generally require less computational resources to train, making them suitable for large datasets. However, these methods are based on a set of assumptions about the underlying distribution of the data, which may not always hold in practice. As a result, parametric data synthesizers may not accurately capture complex relationships between variables, such as interactions and non-linearities.

On the other hand, non-parametric data synthesizers, such as decision tree-based methods, offer greater flexibility and the ability to handle complex relationships between variables. These methods are able to overcome collinearity issues by partitioning the data into subsets and considering the relationships between variables within each subset. However, non-parametric data synthesizers can be computationally expensive and have a higher model complexity, making them less suitable for large datasets.

In terms of random forest-based bagging data synthesizer, it has the advantages of both parametric and non-parametric data synthesizers. The use of multiple decision trees in the random forest model provides a robust way to capture complex relationships between variables, while the bagging technique helps to reduce the variance in the model and overcome collinearity issues. However, the cost of training a random forest-based bagging synthesizer can be high, and its implementation may be more complex than that of parametric data synthesizers.

In summary, the choice of data synthesizer will depend on the specific requirements of the data and the goals of the analysis. While parametric data synthesizers are simpler and more efficient, non-parametric data synthesizers offer greater flexibility and the ability to handle complex relationships between variables. The random forest-based bagging data synthesizer offers a trade-off between these two approaches, making it applicable in a lot of data synthesis tasks.

\subsubsection{Generative Networks with the Exponential Mechanism (GEM)}
\label{subsubsec:terrance}
GEM (Generative Networks with the Exponential Mechanism) is introduced by \citet{liu2021iterative} while working on private query release. It optimizes over past queries to improve accuracy by training a generator network $G_{\theta}$ to implicitly learn a distribution of the data domain, where $G_{\theta}$ can be any neural network parametrized by weights $\theta$. As a result, the GEM method can compactly represent a distribution for any data domain while enabling fast, gradient-based optimization via auto-differentiation frameworks like Pytorch and TensorFlow \citep{paszke2019pytorch,abadi2016tensorflow}.

Concretely, $G_{\theta}$ takes random Gaussian noise vectors z as input and outputs a representation $G_{\theta}(z)$ of a product distribution over the data domain. Specifically, this product distribution representation takes the form of a $d^{'}-$dimensional probability vector $G_{\theta}(z) \in [0,1]^{d^{'}}$, where $d^{'}$ is the dimension of the data in one-hot encoding and each coordinate $G_{\theta}(z)_j$ corresponds to the marginal probability of a categorical variable taking on a specific value. To obtain this probability vector, we choose softmax as the activation function for the output layer in $G_{\theta}$. Therefore, for any fixed weights $\theta$, $G_{\theta}$ defines a distribution over $\mathcal{X}$ through the generative process that draws a random $z \sim \mathcal{N}(0, \sigma ^2I)$ and then outputs random $x$ drawn from the product distribution $G_{\theta}(z)$. We will denote this distribution as $P_\theta$.

To define the loss function for GEM, we require that it be differentiable so that we can use gradient-based methods to optimize $G_{\theta}$. Therefore, we need to obtain a differentiable variant of $q$. Given that a query is defined by some predicate function $\phi:\mathcal{X}\rightarrow \{0,1\}$ over the data domain X that evaluates over a single row $x \in \mathcal{X}$ . We observe then that one can extend any statistical query $q$ to be a function that maps a distribution $P_\theta$ over $\mathcal{X}$ to a value in $[0, 1]$:
\begin{align}
    \label{equ:qfunction}
    q(P_{\theta})=\mathbb{E}_{x\sim p_\theta}[\phi(x)]=\sum _{x\in\mathcal{X}}\phi(x)P_\theta(x).
\end{align}
Note that any statistical query $q$ is then differentiable w.r.t. $\theta$:
\begin{align*}
    \nabla_\theta[q(P_\theta)]=\sum_{x\in \mathcal{X}}\nabla_\theta P_\theta(x)\phi(x)=\mathbb{E}_{\textbf{z}\sim N(0,I_k)}\left [ \sum_{x\in \mathcal{X}}\phi(x)\nabla_\theta\left [ \frac{1}{k}\sum_{i}^{k}\prod_{j=1}^{d'}(G_\theta(z_i)_j)^{x_j}  \right ] \right ],
\end{align*}
and we can compute stochastic gradients of $q$ w.r.t. $\theta$ with random noise samples $\textbf{z}$. This also allows us to derive a differentiable loss function in the Adaptive Measurements framework. In each round $t$, given a set of selected queries $\tilde{Q}_{1:t}$ and their noisy measurements $\tilde{A}_{1:t}$ GEM minimizes the following $l1-$loss:
\begin{align}
    \label{equ:l1-loss}
    \mathcal{L}^{GEM}\left ( \theta,\tilde{Q}_{1:t}, \tilde{A}_{1:t}\right )=\sum_{i=1}^{t}\left | \tilde{q_i}(P_\theta)- \tilde{a}_{i}\right |,
\end{align}
with $\tilde{q_i}\in \tilde{Q}_{1:t}$ and $\tilde{a_i}\in \tilde{A}_{1:t}$.

In general, $\mathcal{L}^{GEM}$ is optimized by running stochastic (sub)-gradient descent. However, it is worth noting that gradient computation can be expensive since obtaining a low-variance gradient estimate often requires calculating $\nabla_\theta[q(P_\theta)]$ for a large number of $x$. 

For query with many classes, however, there exists some closed-form, differentiable function surrogate to \eqref{equ:l1-loss} that evaluates $q(G_\theta(z))$ directly without operating over all $x \in \mathcal{X}$. Concretely, we say that for certain query classes, there exists some representation $f_{q}:\Delta (\mathcal{X})\rightarrow [0,1]$ for$ q$ that operates in the probability space of $\mathcal{X}$ and is also differentiable.

In their work \citep{liu2021iterative}, GEM is implemented to answer $k-$way marginal queries, which have been one of the most important query classes and provides a differentiable form when extended to be a function over distributions. Let the data universe with $d$ categorical attributes be $(\mathcal{X}_1 \times \cdots \times \mathcal{X}_d)$, where each $\mathcal{X}_i$ is the discrete domain of the $i-$th attribute $A_i$. A $k-$way marginal query is defined by a subset $S\subseteq [d]$ of $k$ features (i.e., $|S| = k$) plus a target value $y\in\prod_{i\in S}\mathcal{X}_i$ for
each feature in $S$. Then the marginal query $\phi_{S,y}(x)$ is given by:
\begin{align}
    \label{equ:phi}
    \phi_{S,y}(x)=\prod _{i\in S}\mathbb{1}(x_i=y_i),
\end{align}
where $x_i\in\mathcal{X}_i$ means the $i-$th attribute of record $x\in \mathcal{X}$. Each marginal has a total of $\prod_{i=1}^{k}|\mathcal{X}_i|$ queries, and we define a workload as a set of marginal queries. We consider algorithms that input a dataset $P$ and produce randomized outputs that depend on the data. The output of a randomized mechanism $\mathcal{M}:\mathcal{X}^{\ast}\rightarrow \mathcal{R}$, is a privacy preserving computation if it satisfies differential privacy (DP), given by \citet{dwork2006calibrating}. We say that two datasets are neighboring if they differ in at most the data of one individual.

In particular, it is shown that $k-$way marginals can be rewritten as differentiable product queries.

Let $p \in \mathbb{R}^{d^\prime}$ be a representation of a dataset (in the one-hot encoded space), and let $S \subseteq [d^\prime]$ be some subset of dimensions of $p$. Then we can define a product query $f_S$ as
\begin{align}
    \label{equ:productquery}
    f_S(p)=\prod_{j\in S}p_j.
\end{align}
A $k-$way marginal query $\phi$ can then be rewritten as \eqref{equ:productquery}, with $p=G_\theta (z)$ and $S$ being the subset of dimensions of
corresponding to the attributes $A$ and target values $y$ specified by $\phi$, shown with \eqref{equ:phi}. Thus, any marginal query can be rewritten $\prod_{j\in S}G_\theta(z)_j$, which is differentiable w.r.t. $G_\theta$ (and therefore differentiable w.r.t weights $\theta$ by chain rule). Gradient-based optimization techniques can then be used to solve \eqref{equ:l1-loss}; the exact details of our implementation can be found in Appendix \ref{subsec:gem}.





\newpage

\section{Utility Evaluation and Risk Assessment}
\label{chapter4:eva}
Given we have explained the detailed data synthesis algorithms in section \ref{chapter3:syn}, now we will focus on the data utility and risk assessment of synthetic data generated using parametric and non-parametric data synthesizers. In particular, we will discuss different data utility evaluation metrics suitable for different application scenarios. Besides that, approaches to measure the risk disclosure for both fully and partially synthetic data will be followed. By providing the evaluation metric in terms of utility and risk assessment, we aim to provide a basis for selecting an appropriate data synthesis algorithm for a specific use case, and for evaluating the trade-offs between utility and risk to ensure the quality on the synthetic dataset generated.

\subsection{Data Utility Evaluation}
\label{subsec:utility}
In the realm of evaluating the validity of data that has been altered to maintain confidentiality, a substantial amount of research has been conducted. A majority of these methods can also be utilized to assess the validity of synthetic data. In this chapter, we will concentrate on the measures that are particularly pertinent for synthetic data. For further examination, readers may refer to \citet{hundepool2012statistical}, which delves into additional evaluation measures.

Utility metrics, used to assess the effectiveness of data protection techniques, can be classified into three categories: global, outcome-specific, and fit-for-purpose. Global utility metrics assess the utility of the protected data by directly comparing it to the original data. These metrics have the advantage of not requiring any prior assumptions regarding the intended use of the synthetic data, however, as the utility is measured on a broad level, it does not necessarily ensure high utility for a specific analysis. In contrast, outcome-specific utility metrics measure the utility of the synthetic data for a specific analysis, such as the results of a linear regression model. Fit-for-purpose measures, such as graphical comparisons of the marginal and bivariate distributions of all variables, or consistency checks to prevent implausible values, form the initial stage of any utility assessment.
\subsubsection{Global Utility Metrics}
\label{subsubsec:global}
Utility assessment of synthetic data can be performed by comparing it directly to the original data. A common approach for this comparison is the utilization of distance measures, such as the Kulback-Leibler divergence \citep{karr2006framework} or Hellinger distance \citep{gomatam2003distortion}. However, computing these measures for large datasets can be challenging. An alternative approach, inspired by the literature on propensity score matching \citep{rosenbaum1983central}, aims to evaluate the distinguishability between the original and synthetic data. Propensity scores are estimated by stacking the $n_{org}$ original records and the $n_{syn}$ synthetic records and adding an indicator variable, with a value of 1 indicating the record is from the synthetic data and $0$ otherwise. Then, a model is fitted using the stacked data to estimate the propensity scores, which represent the probability of each record belonging to the synthetic data. If the synthetic data was an exact copy of the original data, it would not offer any information to distinguish between the two datasets, and the distribution of the estimated propensity scores would be the same. One way to measure the utility of the synthetic data is to evaluate the difference in the distribution of the propensity scores between the original and synthetic data. There are various metrics that can be used for this purpose, including the Kolmogorov-Smirnov distance, which is suggested by \citet{mckay2018differentially} and referred to as SPECKS (Synthetic data generation; Propensity score matching; Empirical Comparison via the Kolmogorov-Smirnov distance), or the Mann–Whitney U test (Wilcoxon rank-sum test).

In recent years, the propensity score mean squared error ($p_{MSE}$) has emerged as a widely used metric for evaluating the validity of synthetic data \citep{woo2009global,snoke2018general}. The $p_{MSE}$ is calculated by considering the predicted values obtained from a model, fit on the stacked dataset containing both original and synthetic data, with the variable $c$ equal to $n_{syn}/N$, where $N = n_{org} + n_{syn}$ is the total number of records, which can be formatted as $\frac{1}{N} \sum_{i=1}^{N}(p_i-c)^2$. A lower $p_{MSE}$ value indicates a higher analytical validity of the synthetic data, as $p_i$ approaches $c$ when the model is unable to differentiate between the original and synthetic data, where $p_{i}$ stands for the predicted values ($i=1,...,N$). However, as noted by \citet{woo2009global}, the $p_{MSE}$ may increase with the number of predictors used in the model. To address this issue, \citet{snoke2018general} derived the expected value and standard deviation of the $p_{MSE}$ under the assumption that the synthesis model is correctly specified, and introduced two additional utility measures. The first measure is the $p_{MSE}$ ratio, which is the ratio of the empirical $p_{MSE}$ to its expected value under the null hypothesis. The second measure is the standardized $p_{MSE}$, a.k.a $Sp_{MSE}$, which is calculated as the difference between the empirical $p_{MSE}$ and its expected value under the null, divided by its standard deviation under the null.



\subsubsection{Outcome-specific Utility Measures}
\label{subsubsec:outcome}
The focus of these measures is explicitly on evaluating the suitability of the synthetic data for a designated analysis task. One simple way to assess the analytical validity is to graph the estimates obtained from the original data, such as means or regression coefficients, against the corresponding estimates obtained from the synthetic data. If the utility of the synthetic data is high, these coefficients should display a clustering pattern around the $45-$degree line.

The evaluation method discussed previously does not consider the inherent uncertainty of the estimates, which can be significant in certain situations. For instance, larger deviations between the estimates might be acceptable when the sampling error is high, such as when the estimate of interest is based on a small subset of the data. Conversely, the same deviation might not be acceptable if the statistic is based on the entire sample. To account for the uncertainty of the estimates, a widely adopted measure is the confidence interval overlap measure proposed by \citet{karr2006framework}. This measure assesses the average relative overlap between the confidence interval derived from the original data and the confidence interval derived from the synthetic data. A high overlap measure, close to $1$, indicates that the same inferential conclusions can be drawn, regardless of whether the analysis is based on the synthetic data or the original data.

The recent surge in the use of machine learning approaches has also led to the increased popularity of the machine learning efficacy metric, particularly in computer science literature. This type of utility measure, commonly referred to as model comparability measure, assesses the similarity in results obtained from machine learning models trained on synthetic data compared to models trained on original data. The evaluation procedure typically involves training the models of interest on both synthetic and original data, and then comparing the performance of these models based on the same set of test records obtained from the original data. If the commonly used evaluation criteria such as accuracy, $F1-$score, etc., are similar regardless of whether the models were trained on synthetic or original data, the utility of the synthetic data is considered high. Additionally, utility is sometimes evaluated by determining whether the use of synthetic data for model training leads to the same ranking of various machine learning models. For instance, if the original data suggests that a multi-layer perceptron (MLP) classifier outperforms a random forest, which in turn outperforms logistic regression, the same ranking should be obtained when the synthetic data is used for model training.
\subsubsection{Fit-for-purpose Measures}
\label{subsubsec:purpose}
The fit-for-purpose measures represent a preliminary evaluation when determining the utility of the generated data. They are distinguished from the other two measures as they do not necessarily concentrate on evaluating the validity of analyses deemed important for the data users. Nor do they aim to directly assess the similarity of the original and synthetic data through a single global statistic. The primary objective of these measures is to provide a preliminary evaluation of the quality of the synthetic data and to identify any areas of the synthesis process that may require further improvement. This category of measures can be classified into three categories: graphical evaluations, feasibility assessments, and calculation of various goodness-of-fit (GOF) measures.

Graphical evaluations in the assessment of the usefulness of generated data constitute a key strategy and include tactics such as the presentation of side-by-side comparisons of the marginal distributions of both the synthetic and original data, as well as contour plots for the examination of bivariate distributions. Additionally, visual examinations of conditional distributions, such as the income distribution across different demographic categories, can also be performed.

Plausibility checks, on the other hand, necessitate the involvement of subject-matter experts who are familiar with the data, as not all discrepancies can be readily discerned. For instance, while identifying anomalies such as the existence of two-year-old married individuals may be straightforward, determining the credibility of changes in annual turnover for a particular establishment within a specific industry may prove more challenging.

Finally, any GOF measure can be utilized to gauge the similarity between specific aspects of the synthetic and original data. For example, the Kolmogorov-Smirnov test statistic can be employed to assess the similarity of continuous variables in the dataset, while cross-tabulations of several variables can be evaluated through the utilization of the $\chi^2$ statistic or the likelihood ratio statistic. While the test statistic should not be used to test for statistically significant differences between the synthetic and original data, as the two samples cannot be treated as independent due to the synthetic data being generated based on information from the original data, the value of the test statistic can still be utilized to evaluate the efficacy of different synthesis techniques and to identify potential issues with the quality of the synthetic data. For instance, a high test statistic for many cross-tabulations involving the age variable may indicate that further improvement is required in the synthesis of this variable.

The $p_{MSE}$ measure described in Section \ref{subsubsec:global} can serve as a fit-for-purpose measure by only considering the relevant variables when computing the propensity score. \citet{raab2021assessing} provides an example of how this approach can be utilized to visualize the utility for bivariate distributions. Additionally, the R package "synthpop" implements graphical visualization tools for this purpose.

In the study conducted by \citet{raab2021assessing}, the authors investigated the correlation between various goodness-of-fit measures and found a high correlation ($>0.9$) among most of them. Notably, a correlation above $0.99$ was found between the adjusted $\chi^2$ test proposed by \citet{voas2001evaluating}, the Freeman-Tukey statistic, the Jensen-Shannon divergence (JSD), and the $p_{MSE}$, as well as between the Kolmogorov-Smirnoff test statistic, the Mann-Whitney test statistic, and two additional measures that were not further discussed for brevity. This result suggests that it is sufficient to utilize only one or two GOF criteria when evaluating the utility of the generated data.


\subsection{Risk Disclosure Analysis}
\label{subsec:risk}
The distinction between partial and full synthesis of data poses different risks from a disclosure protection perspective. With partial synthesis, there still exists a one-to-one correspondence between the original and the synthetic data. On the other hand, in full synthesis, such correspondence no longer holds, as the synthetic data do not necessarily have to be of the same size as the original data. This makes it difficult to measure the risk of re-identification, as commonly done for other disclosure protection methods \citep{reiter2005estimating, skinner2008assessing, shlomo2014probabilistic}. However, it does not imply that fully synthetic data cannot pose any risk of leaking sensitive information.

\citet{manrique2018bayesian} demonstrate using real data that when a fully conditional specification approach (a method commonly applied in multiple imputation for nonresponse) is utilized for synthesis based on CART, there is a risk that the synthesizer will replicate most of the original records. This occurs because the approach always conditions on all other variables in the dataset, leading to situations where the synthesized variable values are completely determined by the other variables in complex datasets containing many categorical variables. The CART synthesizer may become trapped in this situation, resulting in replication of records from the original data. Such a problem can be avoided by not using the fully conditional specification approach, which offers no advantages in the context of synthetic data.

However, this example highlights that fully synthetic data cannot be assumed to be free from the risk of disclosure of sensitive information. While measuring these risks is challenging, research in this area remains limited.

In this section, we initiate the process of reviewing the literature on the methods proposed to evaluate the risks associated with fully synthetic data. These evaluations can also be applied to partially synthetic data, while the risk assessments reviewed in the subsequent part of this section are only applicable to partial synthesis and aim to assess the risk of re-identification for the generated data.

\subsubsection{Risk Assessment for Fully Synthetic Data}
\label{subsubsec:riskfullsyn}
This section examines the methods that have been put forth in the literature for evaluating the risks associated with fully synthetic data. Although the connection between the original and synthetic data is severed in the case of full synthesis, some organizations still assess how many synthetic records have a one-to-one correspondence with the original data. This evaluation is driven by the perceived risk that survey respondents may be apprehensive if they find a synthetic record that precisely corresponds with their own record, particularly if their combination of attributes makes them unique in the original data.

Additionally, authors such as \citet{park2018data} and \citet{zhao2021ctab} have computed the distance between the synthetic data records and their nearest neighbors in the original data. The average of these distances across all synthetic records is then employed as a measure of risk. From a practical standpoint, the meaning of this risk measure is not clear. Even if the average distance is small, the distance for some records may be substantial. An attacker would not be able to determine which records have small distances, and a small distance does not necessarily indicate a risk if the closest record is in a densely populated area of the data distribution.

Another risk evaluation measure, proposed by \citet{taub2018differential}, matches cases from the original and synthetic data. The authors divide the variables in the dataset into key variables, which are assumed to be known by the attacker, and target variables, which the attacker is attempting to infer. They assume that the attacker focuses on records with low $l-$diversity for the target variables within a given equivalence class, defined by the key variables. Let $K$ represent the vector containing the key variables and $T$ represent the vector of target variables. The authors define the Within Equivalence Class Attribution Probability (WEAP) as follows:
\begin{align*}
    \label{equ:weap}
    WEAP_{j}=Pr(T_{j}|K_j)=\frac{\sum_{i=1}^{n}I(T_i=T_j,K_i=K_j)}{\sum_{i=1}^{n}I(K_i=K_j)},
\end{align*}
in which, the indicator function, I(·), equals to 1 when the statement inside the parenthesis is true and equals to 0 when it is false. The value of n refers to the size of the database. The authors in \citet{taub2018differential} concentrate on the synthetic records for which the value of $WEAP_j$ is equal to 1, and for those records, they calculate the Targeted Correct Attribution Probability (TCAP) with the following equation:
\begin{align*}
    \label{equ:tcap}
    TCAP_{sj}=Pr(T_{sj}|K_{sj})_{o}=\frac{\sum_{i=1}^{n}I(T_{o,i}=T_{s,j},K_{o,i}=K_{s,j})}{\sum_{i=1}^{n}I(K_{o,i}=K_{s,j})},
\end{align*}
with the subscript $s$ representing synthetic data and $o$ representing the original data. Note that the Targeted Correct Attribution Probability (TCAP) score is defined in a range between zero and one, with higher values denoting greater risk.

Another category of risk assessments for fully synthetic data concentrates on the fact that the synthesis models themselves may inadvertently expose information regarding the contents of the original data. For instance, the use of a fully saturated log-linear model in conjunction with vague prior information to synthesize a set of categorical variables may reveal the presence of specific attribute combinations in the synthetic data, implying that the same combinations must have been present in the original data. These types of risk assessments, referred to as membership attacks in the computer science literature, occur when an attacker gains knowledge that a particular record was included in the original data. The literature presents various methods for estimating the risks associated with membership attacks, which typically presume that the attacker has prior knowledge of the true values for some target records, using this information to determine whether these units are present in the original data \citep{stadler2022synthetic}. These assessments are premised on the assumption that the attacker is not interested in acquiring new information about a unit contained within the data, but only seeks to determine whether the unit was part of the original data. In some cases, learning this information may be deemed unacceptable, as certain laws specify that such risks must be avoided, and knowledge of inclusion in a database may also expose sensitive information, particularly if the database only contains a specific subset of the population, as seen in the example of the Survey of Prison Inmates conducted by the Bureau of Justice Statistics in the United States.

Despite the challenges involved, there exist risk measures for fully synthetic data that do not make the assumption that the attacker has knowledge of the original data. In the realm of inferential attacks, \citet{reiter2014bayesian} proposes a method for computing the posterior distribution $f(Y_i |D,X,M,d_{org}^{-i})$, where $Y_i$ represents the original value of a variable $Y$ for a unit $i$, $D$ represents the synthetic data, $X$ contains unchanged values from the original data (if full synthesis is employed, $X$ is empty), $M$ contains information regarding the synthesis model, and $d_{org}^{-i})$ denotes the original data excluding the record $i$. This method evaluates how much an attacker can learn about an unknown value $Y_i$ based on the synthetic data. If the posterior distribution of $Y_i$ exhibits low variability in comparison to the prior distribution prior to observing the synthetic data, the risk of disclosure is high. While the strong assumption that the attacker knows all information from the original data excluding record $i$ is not strictly necessary, it is often required for the purpose of making the problem computationally feasible. Nevertheless, even with these assumptions, this risk assessment method is only applicable when the number of variables in the data is very limited, as demonstrated in \cite{hu2014disclosure}.

In conclusion, assessing disclosure risks for fully synthetic data remains a complex challenge, and while the majority of researchers acknowledge that these data are not immune to risks, additional research is required to quantify these risks under realistic conditions.



\subsubsection{Risk Assessment for Partially Synthetic Data}
\label{subsubsec:riskpartialsyn}
In the context of partial synthesis, the risk measures discussed previously can also be applied. However, the existence of a unique match between the synthetic and original records implies that the risk assessment should focus on the potential for an attacker to re-identify records in the synthetic data. Building upon the work of \citet{reiter2005estimating}, \citet{reiter2009estimating} proposed methods to evaluate the re-identification risk in partially synthetic data.

Building on the work of \citet{drechsler2010sampling}, the process of evaluating the risk of re-identification for partially synthetic data can be summarized as follows. Given the information an intruder possesses on a specific target unit in the population, denoted by $t$, where $t_0$ represents the target's unique identifier, and $P_{i0}$ represents the unique identifier for the $i$th record in the synthetic data ($d_{syn}$), the intruder aims to determine the match between the target and the $i-$th record in $d_{syn}$ such that $P_{i0}= t_0$. The variable $J$ is a random variable that takes the value of $i$ when $P_{i0} = t_0$ for $i$ in $d_{syn}$. The intruder seeks to compute the probability $Pr(J = i|t, d_{syn}, S)$ for $i = 1, . . . , n$, where $S$ represents any information released about the synthesis models.

Since the intruder is not aware of the actual values of the synthesized variable ($Y^{∗}$), the computation of the match probabilities must involve integrating over all possible values of $Y^{∗}$. The following formula is then calculated for each record:
\begin{align}
    \label{equ:risk-partial}
    Pr(J = i|t, d_{syn}, S)=\int Pr(J = i|t, d_{syn}, Y^*,S)Pr(Y^*|t, d_{syn}, S)dY^*.
\end{align}
The risk computations shown in equation \eqref{equ:risk-partial} for partially synthetic data can be estimated through a Monte Carlo approach as proposed by \citet{drechsler2010sampling}. The intruder seeks to determine the probability, $Pr(J = i|t, d_{syn}, S)$, of matching unit $i$ in the synthetic data, $d_{syn}$, to a particular target unit in the population, $P$, based on information $t$ and the released information $S$ about the synthesis models. To estimate this probability, the intruder first samples a value of the synthesized variable, $Y^*$, from the probability distribution, $Pr(Y^∗|t,d_{syn},S)$. This value is used in conjunction with the matching strategy, such as nearest neighbor matching, to compute the match probability. This two-step process is repeated a large number of times and the expected match probability is estimated as the average of these computed probabilities.

The disclosure risk can be measured using three risk measures proposed by \citet{reiter2009estimating}: the expected match risk, the true match rate, and the false match rate. The expected match risk is the sum of the number of true matches divided by the number of records with the highest match probability for the target. The true match rate is the percentage of true unique matches among the target records. The false match rate is the percentage of false matches among unique matches.

It should be noted that these risk assessments assume that the intruder knows that the target record is included in the released data. \citet{drechsler2008accounting} present extensions of this approach that also account for the additional uncertainty from sampling when the intruder does not know whether the target participated in the survey.




% \subsection{Analysis-specific Utility from Regression Models}
% \label{subsec:fit}
% ...

% \subsubsection{Analysis from Fitted Linear Regression Models}
% \label{subsubsec:fit-lm}

% \subsubsection{Analysis from Fitted Generalized Linear Regression Models}
% \label{subsubsec:fit-glm}


\newpage

\section{Empirical Evaluation of Synthetic Data generated from CTIS}
\label{chapter5:ctis}
Chapter 5 covers the generation and evaluation of synthetic datasets using both parametric and non-parametric data synthesizers, as mentioned in section \ref{chapter3:syn}. Firstly, this chapter introduces the data source, which is obtained from the University of Maryland Social Data Science Center Global COVID-19 Trends and Impact Survey in collaboration with Facebook. Then, the necessary data preprocessing for metadata is discussed, in order to establish a more organized experiment structure when dealing with variables. Thirdly, this chapter covers the detailed experiment settings, including the design of synthesizing methods, orders, and workflow. Following this, this chapter interprets and explains the results obtained from the synthetic datasets, encompassing data utility evaluation and risk disclosure analysis. Additionally, the chapter compares the inference from fitted linear models for both original and synthetic datasets. Finally, this chapter also lists other findings, such as the effect of model complexity on synthesizing quality and efficiency, and synthesizing with missing data.

\subsection{The CTIS Dataset}
\label{subsec:ctis}
Before going through the detailed preprocessing of variables, we present a table which shows a list of variables included in the CTIS datasets, including the variable naming, question text, and recorded response values.
\begin{longtblr}[
  caption = {Long Title},
  label = {tab:allvars},
]{
  colspec = {X[1.5]X[5]X[2]},
  rowhead = 1,
  % hlines,
  % row{even} = {gray9},
  %row{1} = {white9},
} 
\hline
\textbf{Variable} & \textbf{Question text} & \textbf{Responses}\\\hline
survey\_region  & There are two versions of the survey. & EU = European Union\\
&The only difference is the initial consent statement.&ROW = Rest of World\\\hline
weight  & survey weight to adjust from FB user population to the general population &number(float)\\\hline
Finished  & Qualtrics metadata indicating whether completed the entire questionnaire &1 = yes, 0 = no\\\hline
RecordedDate  & Date that the response was recorded. &date/time\\\hline
\multicolumn{3}{l}{\textsc{SECTION B}}&&\\\hline
B1\_1  & In the last 24 hours, have you had any of the following? Fever &1=Yes, 2=No\\\hline
B1\_2  & In the last 24 hours, have you had any of the following? Cough &1=Yes, 2=No\\\hline
B1\_3  & In the last 24 hours, have you had any of the following? Difficulty breathing &1=Yes, 2=No\\\hline
B1\_4  & In the last 24 hours, have you had any of the following? Fatigue &1=Yes, 2=No\\\hline
B1\_5  & In the last 24 hours, have you had any of the following? Stuffy or runny nose &1=Yes, 2=No\\\hline
B1\_6  & In the last 24 hours, have you had any of the following? Aches or muscle pain &1=Yes, 2=No\\\hline
B1\_7  & In the last 24 hours, have you had any of the following? Sore throat &1=Yes, 2=No\\\hline
B1\_8  & In the last 24 hours, have you had any of the following? Chest pain &1=Yes, 2=No\\\hline
B1\_9  & In the last 24 hours, have you had any of the following? Nausea &1=Yes, 2=No\\\hline
B1\_10  & In the last 24 hours, have you had any of the following? Loss of smell or taste &1=Yes, 2=No\\\hline
B1\_11  & In the last 24 hours, have you had any of the following? Eye pain &1=Yes, 2=No\\\hline
B1\_12  & In the last 24 hours, have you had any of the following? Headache &1=Yes, 2=No\\\hline
B1\_13  & In the last 24 hours, have you had any of the following? Chills &1=Yes, 2=No\\\hline
B1\_14  & In the last 24 hours, have you had any of the following? Changes to sleep &1=Yes, 2=No\\\hline
B2  & For how many days have you had at least one of these symptoms? Cough &\textsc{OPEN RESPONSE: NUMBER VALIDATION}\\\hline
B1b\_x1  & Are any of these symptoms unusual for you? Fever &1=Yes, 2=No\\\hline
B1b\_x2  & Are any of these symptoms unusual for you? Cough &1=Yes, 2=No\\\hline
B1b\_x3  & Are any of these symptoms unusual for you? Difficulty breathing &1=Yes, 2=No\\\hline
B1b\_x4  & Are any of these symptoms unusual for you? Fatigue &1=Yes, 2=No\\\hline
B1b\_x5  & Are any of these symptoms unusual for you? Stuffy or runny nose &1=Yes, 2=No\\\hline
B1b\_x6  & Are any of these symptoms unusual for you? Aches or muscle pain &1=Yes, 2=No\\\hline
B1b\_x7  & Are any of these symptoms unusual for you? Sore throat &1=Yes, 2=No\\\hline
B1b\_x8  & Are any of these symptoms unusual for you? Chest pain &1=Yes, 2=No\\\hline
B1b\_x9  & Are any of these symptoms unusual for you? Nausea &1=Yes, 2=No\\\hline
B1b\_x10  & Are any of these symptoms unusual for you? Loss of smell or taste &1=Yes, 2=No\\\hline
B1b\_x11  & Are any of these symptoms unusual for you? Eye pain &1=Yes, 2=No\\\hline
B1b\_x12  & Are any of these symptoms unusual for you? Headache &1=Yes, 2=No\\\hline
B1b\_x13  & Are any of these symptoms unusual for you? Chills &1=Yes, 2=No\\\hline
B1b\_x14  & Are any of these symptoms unusual for you? Changes to sleep &1=Yes, 2=No\\\hline
B3 &Do you personally know anyone in your local community who is sick with a fever and either a cough or difficulty breathing?& 1=Yes, 2=No\\\hline
B4 &How many people do you know with these symptoms?& \textsc{OPEN RESPONSE: NUMBER VALIDATION}\\\hline
B5 &Have you spent time with any of these people in the last 7 days?& 1=Yes, 2=No\\\hline
B6 &Have you ever been tested for coronavirus (COVID-19)?& 1=Yes, 2=No\\\hline
B7 &Have you been tested for coronavirus (COVID-19) in the last 14 days?& 1=Yes, 2=No\\\hline
B8 &Did your most recent test find that you had coronavirus (COVID-19)?& 1=Yes, 2=No, 3=I don't know\\\hline
B9 &Did you have to pay anything out-of-pocket for this test& 1=Yes, 2=No, 3=I don't know\\\hline
B10 &Have you or your household had to reduce spending on things you need (such as food, housing, or medication) because of the cost you paid to get the coronavirus (COVID-19) test?& 1=Yes, 2=No, 3=I don't know\\\hline
B11 &Have you wanted to get tested for coronavirus (COVID-19) at any time in the last [feed days back - cap at 14] days?& 1=Yes, 2=No\\\hline
B12\_1  & Do any of the following reasons describe why you haven't been tested for coronavirus (COVID-19) in the last [feed days back - cap at 14] days? [y/n] I tried to get a test but was not able to get one &1=Yes, 2=No\\\hline
B12\_2  & Do any of the following reasons describe why you haven't been tested for coronavirus (COVID-19) in the last [feed days back - cap at 14] days? [y/n] I don't know where to go &1=Yes, 2=No\\\hline
B12\_3  & Do any of the following reasons describe why you haven't been tested for coronavirus (COVID-19) in the last [feed days back - cap at 14] days? [y/n] I can't afford the cost of the test &1=Yes, 2=No\\\hline
B12\_4  & Do any of the following reasons describe why you haven't been tested for coronavirus (COVID-19) in the last [feed days back - cap at 14] days? [y/n] I don't have time to get tested &1=Yes, 2=No\\\hline
B12\_5  & Do any of the following reasons describe why you haven't been tested for coronavirus (COVID-19) in the last [feed days back - cap at 14] days? [y/n] I am unable to travel to a testing location (including because of transportation cost, safety, or physical limitations) &1=Yes, 2=No\\\hline
B12\_6  & Do any of the following reasons describe why you haven't been tested for coronavirus (COVID-19) in the last [feed days back - cap at 14] days? [y/n] I am worried about bad things happening to me or my family (including discrimination, government policies, and social stigma) &1=Yes, 2=No\\\hline
B13\_1  & In the last 30 days, has there been any time when you needed any of the following health services or products but could not get it? Emergency transportation services or emergency rescue &1=Yes, 2=No\\\hline
B13\_2  & In the last 30 days, has there been any time when you needed any of the following health services or products but could not get it? Medical care with overnight stay in any type of facility &1=Yes, 2=No\\\hline
B13\_3  & In the last 30 days, has there been any time when you needed any of the following health services or products but could not get it? Medical or dental care or treatment without an overnight stay&1=Yes, 2=No\\\hline
B13\_4  & In the last 30 days, has there been any time when you needed any of the following health services or products but could not get it? Preventative health services (including immunization/vaccination, family planning, prenatal/postnatal care, routine check-up services) &1=Yes, 2=No\\\hline
B13\_5  & In the last 30 days, has there been any time when you needed any of the following health services or products but could not get it? Medication &1=Yes, 2=No\\\hline
B13\_6  & In the last 30 days, has there been any time when you needed any of the following health services or products but could not get it? Mask, medical gloves, or other protective equipment &1=Yes, 2=No\\\hline
B13\_7  & In the last 30 days, has there been any time when you needed any of the following health services or products but could not get it? Eyeglasses, hearing aid, crutches, band-aids/plasters, thermometer, or any other health product &1=Yes, 2=No\\\hline
B14\_1  & In the last 30 days, have you been unable to get needed treatment, services, medicine, or medical products for any of the following reasons? I didn't know where to go &1=Yes, 2=No\\\hline
B14\_2  & In the last 30 days, have you been unable to get needed treatment, services, medicine, or medical products for any of the following reasons? I couldn't afford the treatment, service, or product &1=Yes, 2=No\\\hline
B14\_3  & In the last 30 days, have you been unable to get needed treatment, services, medicine, or medical products for any of the following reasons? I was unable to travel to the health care provider (including because of transportation cost, safety, or physical limitations) &1=Yes, 2=No\\\hline
B14\_4  & In the last 30 days, have you been unable to get needed treatment, services, medicine, or medical products for any of the following reasons? I was afraid of being infected at the health care provider &1=Yes, 2=No\\\hline
B14\_5  & In the last 30 days, have you been unable to get needed treatment, services, medicine, or medical products for any of the following reasons? The treatment, service, or product was not available &1=Yes, 2=No\\\hline
\multicolumn{3}{l}{\textsc{SECTION C}}&&\\\hline
C0\_1  & In the last 24 hours, have you done any of the following? Gone to work outside the place where you are currently staying &1=Yes, 2=No\\\hline
C0\_2  & In the last 24 hours, have you done any of the following? Gone to a market, grocery store, or pharmacy &1=Yes, 2=No\\\hline
C0\_3  & In the last 24 hours, have you done any of the following? Gone to a restaurant, cafe, or shopping center &1=Yes, 2=No\\\hline
C0\_4  & In the last 24 hours, have you done any of the following? Spent time with someone who isn't currently staying with you &1=Yes, 2=No\\\hline
C0\_5  & In the last 24 hours, have you done any of the following? Attended a public event with more than 10 people &1=Yes, 2=No\\\hline
C0\_6  & In the last 24 hours, have you done any of the following? Used public transit &1=Yes, 2=No\\\hline
C13\_1  & In the last 24 hours, have you worn a mask when you have done any of the following? Gone to work outside the place where you are currently staying &1=Yes, 2=No\\\hline
C13\_2  & In the last 24 hours, have you worn a mask when you have done any of the following? Gone to a market, grocery store, or pharmacy &1=Yes, 2=No\\\hline
C13\_3  & In the last 24 hours, have you worn a mask when you have done any of the following? Gone to a restaurant, cafe, or shopping center &1=Yes, 2=No\\\hline
C13\_4  & In the last 24 hours, have you worn a mask when you have done any of the following? Spent time with someone who isn't currently staying with you &1=Yes, 2=No\\\hline
C13\_5  & In the last 24 hours, have you worn a mask when you have done any of the following? Attended a public event with more than 10 people &1=Yes, 2=No\\\hline
C13\_6  & In the last 24 hours, have you worn a mask when you have done any of the following? Used public transit &1=Yes, 2=No\\\hline
\end{longtblr}

The quick brown fox jumps over the lazy dog.
The quick brown fox jumps over the lazy dog.

In this study, we utilize the CTIS dataset as the original dataset to synthsize from, which is an abbreviation for the "Covid-19 Trend and Impact Surveys" dataset. The CTIS dataset is obtained from the data repository of The University of Maryland Social Data Science Center Global COVID-19 Trends and Impact Survey, in partnership with Facebook  \citep{salomon2021us}. For the purposes of our analysis, we focus on the microdata recorded from August 2nd to August 8th, 2020. During this time frame, the CTIS dataset is categorized into several sections, including Section B (Health), Section C (Contacts), Section D (Mental Health and Economic Security), Section E (Demographics), Section F (App), as well as other generic variables such as Survey Weight and Recorded Date. These sections contain a wealth of information relevant to the impact of COVID-19 on various aspects of people's lives, and thus provide a valuable source of data for our study. 

Specifically, section B of the CTIS dataset contains several health-related variables, some of which are of particular interest to this study. One such variable is the \textbf{B1\_i} variable, which queries whether respondents have experienced any of a list of symptoms in the last 24 hours, including but not limited to fever, fatigue, chest pain, and headache. Responses to the \textbf{B1\_i} variable are encoded as binary variables, with a value of 1 indicating that the symptom was experienced and 2 indicating the opposite. Another variable of interest is \textbf{B2}, which queries the duration of symptoms experienced by respondents. In the meantime, \textbf{B2} is an open-response variable that differs in numerical validation. Additionally, variables such as \textbf{B1b\_xi} and their corresponding binary variables, \textbf{B1\_i}, also query the presence of symptoms, but with the added dimension of whether the symptoms are unusual for the respondent. These health-related variables provide valuable information on the health status and symptomatology of respondents, which are worth taking into consideration when evaluating the impact of COVID-19 on individuals. Note that i is a integer ranging from 1 to 14.

The following section C comprises variables that pertain to contacts, and is of particular relevance to this study. For instance, the \textbf{C0\_1} variable in this section queries whether respondents have left their current location to go to work outside the home within the last 24 hours. Responses to \textbf{C0\_1} are encoded as binary variables, with a value of 1 indicating that the respondent has gone to work outside the home, and 2 indicating the opposite. Additionally, there are categorical variables in this section with classes greater than 2. One such variable, \textbf{C7}, asks respondents to report how many times they have washed their hands with soap and water or used hand sanitizer in the last 24 hours. The responses to \textbf{C7} are encoded as integers, with a value of 1 indicating 0 times, 2 indicating 1-2 times, 3 indicating 3-6 times, and 4 indicating 7 or more times. These variables provide valuable insights into the contact patterns and hygiene practices of respondents, which are important factors to consider while assessing the impact of COVID-19 on individuals and communities.

Section D encompasses variables that relate to mental health and economic security. One example of a variable in this section is the \textbf{D1} categorical variable, which queries respondents on how frequently they have felt nervous in the past 7 days to the point that nothing could calm them down. Responses to \textbf{D1} are encoded on a scale of 1 to 5, with 1 indicating that the respondent felt nervous all the time, and 5 indicating that they felt none of the time. Additionally, the \textbf{D10} variable has much more numerical validation requirements and asks respondents to indicate their occupation or industry, with responses ranging from 1 to 15. Specifically, the responses to \textbf{D10} correspond to the following occupational categories: agriculture, buying and selling, construction, education, electricity/water/gas/waste, financial/insurance/real estate services, health, manufacturing, mining, personal services, professional/scientific/technical activities, public administration, tourism, transportation, and Other. These mental health and economic security-related variables provide crucial insights into the experiences of individuals during the COVID-19 pandemic and their associated impacts.

Another crucial section of the CTIS dataset is section E, which contains variables related to demographics. For example, the \textbf{E3} variable queries respondents on their gender and provides categorical responses, with 1 indicating male, 2 indicating female, 3 indicating other, and 4 indicating a preference not to answer. In addition to \textbf{E3}, this section also includes open-response variables such as \textbf{E6}, which queries the number of years of education completed by the respondent. The responses to \textbf{E6} vary depending on the numerical value entered. These demographics-related variables provide valuable information on the characteristics of respondents and their potential impact on the spread and impact of COVID-19.

Moving on to Section F, which focuses on app-related variables, we find that it contains several questions regarding the use of contact and symptom tracing apps. Specifically, the \textbf{F2\_1} variable asks respondents whether they have installed a contact tracing app on their smartphone, while the \textbf{F2\_2} variable queries whether they have installed a symptom tracing app. Both of these variables elicit binary responses, with a value of 1 indicating that the respondent has installed the app, and a value of 2 indicating that they have not. These variables provide valuable insights into the use and uptake of contact and symptom tracing apps, which are important tools in controlling the spread of COVID-19.

The final section of the CTIS dataset, Section G, comprises geographic variables that provide valuable information on the location and distribution of respondents. One example of such a variable is \textbf{GID\_0}, which represents countries and is encoded using the ISO-Alpha 3 code. Another variable, \textbf{GID\_1}, provides more detailed regional information by assigning a unique ID to each subdivision of \textbf{GID\_0}. These geographic variables are important for assessing the spatial distribution of COVID-19 cases and for understanding how the pandemic has impacted different regions and countries.

In summary, the CTIS dataset comprises substantial information relevant to the impact of COVID-19 on various aspects of people's lives, including health, contacts, mental health and economic security, demographics, app usage, and geographic location. The dataset includes numerous variables within each section, providing valuable insights into the experiences of individuals during the pandemic.

Due to the large amount of metadata recorded from August 2nd to August 8th, 2020, our study will conduct data preprocessing to remove variables that are redundant and focus on those variables of statistical interest. This approach will allow us to reduce the dimensionality of the dataset and generate synthetic datasets through simulations.

It is important to mention that all the missingness representing for missing answers in the corresponding question variable. For simplicity and data alignment, these missing data are all encoded with value -99 indicating the presence of missingness.





% In the following section, we will introduce the data preprocessing methods employed and explain their rationale for selecting specific variables.


\subsection{Data Preprocessing}
\label{subsec:preprocess}
In the data preprocessing stage, our goal was to prepare the CTIS dataset for use in generating synthetic datasets. Given the large number of variables in each section of the dataset, we first filtered out non-European countries to focus on our region of interest. This was achieved by creating a separate file, "gpdr.csv," containing the names of the European countries of interest, and filtering the original dataset based on the \textbf{GID\_0} variable, which represents countries and is encoded using the ISO-Alpha 3 code.


Following this filtering step, we concatenated the remaining observations vertically based on the date, resulting in a total of 260,299 observations with 92 variables in total. First of all, we remove two region specified variables, i.e. \textbf{GID\_0} and \textbf{GID\_1}, with 90 variables retained. We then reduced the dimensionality of the dataset from 90 to 54 by excluding columns with constant inputs, as well as variables that corresponded to similar questions asked in the corresponding sections. This included variables such as \textbf{B1b\_x1}, \textbf{B1b\_x2}, \textbf{B1b\_x3}, \textbf{B1b\_x4}, \textbf{B1b\_x5}, \textbf{B1b\_x6}, \textbf{B1b\_x7}, \textbf{B1b\_x8}, \textbf{B1b\_x9}, \textbf{B1b\_x10}, \textbf{B1b\_x11}, \textbf{B1b\_x12}, and \textbf{B1b\_x13}, as well as \textbf{C0\_1}, \textbf{C0\_2}, \textbf{C0\_3},\textbf{C0\_4}, \textbf{C0\_5}, and \textbf{C0\_6}. Since the overlapping of defined questions for \textbf{B1b\_xi} variables is covered in section \ref{subsec:ctis}, we only explain for the removal of \textbf{C0\_1}, \textbf{C0\_2}, \textbf{C0\_3}, \textbf{C0\_4}, \textbf{C0\_5}, and \textbf{C0\_6}.  The inquiry posed by the variables \textbf{C0\_1}, \textbf{C0\_2}, \textbf{C0\_3},\textbf{C0\_4}, \textbf{C0\_5}, and \textbf{C0\_6} follows a general format of: "In the last 24 hours, have you done any of the following?". Notably, this question structure resembles that of the \textbf{C13\_1}, \textbf{C13\_2}, \textbf{C13\_3}, \textbf{C13\_4}, \textbf{C13\_5}, and \textbf{C13\_6} variables, which ask: "In the last 24 hours, have you worn a mask when you have done any of the following?". In order to avoid redundancy in questioning, we have excluded the former set of variables (\textbf{C0\_1}, \textbf{C0\_2}, \textbf{C0\_3},\textbf{C0\_4}, \textbf{C0\_5}, and \textbf{C0\_6}) and retained the latter (\textbf{C13\_1}, \textbf{C13\_2}, \textbf{C13\_3}, \textbf{C13\_4}, \textbf{C13\_5}, and \textbf{C13\_6}) for further analysis. Furthermore, we also removed variables that were dependent on the answers from other variables in their respective sections, such as the \textbf{D10} variable, which was dependent on the answers to \textbf{D7} and \textbf{D8}. Specifically, \textbf{D7} asked respondents if they had done any work for pay in the last 7 days, while \textbf{D8} asked if they had been working for pay before February 2020. The answers to these questions were encoded as binary responses, with 1 representing answer=yes and 2 denoting answer=no. By removing unnecessary variables, we were able to reduce the dimensionality of our datasets while retaining a size of 54$\times$260,299. 

The maintained variables and its corresponding sections can be found in table \ref{tab:listofvars}. For the sake of simplicity, we will not provide a detailed list of variable meanings pertained here, please refer to the appendix section \ref{subsec:listofvars}.

\begin{table}[h]
    \centering
    \caption{List of variables included in the synthesis}
    \begin{NiceTabular}{@{}lll@{}}[colortbl-like]\hline 
        \rowcolor{white!90} Section & Variables included \\\hline
        \rowcolor{white!90} Section B & B1\_1, B1\_2, until B1\_13, \\ 
        \rowcolor{white!90} & B2, B3, until B11,  \\ 
        \rowcolor{white!90} & B12\_1, B12\_2 until B12\_6  \\\hline
        \rowcolor{white!90} Section C & C1\_m, C2, C3,\\ 
        \rowcolor{white!90} & C5, C6, C7, C8                                                    \\
        \hline
        \rowcolor{white!90} Section D & D1, D2, until D9                                                                  \\\hline
        \rowcolor{white!90} Section E & E2, E3, E4,\\
        \rowcolor{white!90} & E5, E6, E7                                                           \\
        \hline
        \rowcolor{white!90} Section F & F1, F2\_1, F2\_2                                                                  \\\hline
        % \\[-0.8em]
        \rowcolor{white!90} Other variables & weight  \\\hline
    \end{NiceTabular}
    \label{tab:listofvars}
\end{table}


After the dimensionality reduction techniques mentioned above, we start to discuss further regarding the encoding of variables. In the CTIS dataset, some variables are set with open response questions, which may result in a wide range of possible responses. To reduce this variability in such variables, we have applied an encoding scheme for certain variables, where these open response variables include variable \textbf{B2}, \textbf{B4}, \textbf{E5}, \textbf{E6}. More specifically, the \textbf{B2} variable inquires about the number of days with COVID-19 symptoms experienced by the respondent. To encode the responses for \textbf{B2}, we have used a set of predefined thresholds, which classify the responses into specific ranges. In particular, we have assigned a value of -99 to missing or invalid responses, and values greater than or equal to 1000 have been set to 1000. For the remaining responses, we have categorized them into ranges of increasing values, such as [0,1), [1,3), [3,8),[8,15),[15, 28),[28, 90),[90, 180),[180, 366), until the last category of $[366,1000)$.

Similarly, the \textbf{B4} variable queries how many people the respondent knows with COVID-19 symptoms. To encode this variable, we have employed a similar threshold-based approach. Responses less than 0 are assigned a value of -99, and values greater than or equal to 1000 are set to 1000. The remaining responses are grouped into ranges such as $[0,1), [1,5), [5,10),$ and $[10,1000)$.

The \textbf{E5} variable asks about the number of people who slept in the same place as the respondent on the previous night. As with \textbf{B2} and \textbf{B4}, we have utilized a threshold-based approach to encode responses for \textbf{E5}. Values less than 0 are set to -99, and values greater than or equal to 1000 are set to 1000. The remaining responses are categorized into ranges such as $[0,1), [1,2), [2,4), [4,6)$, and $[6,1000)$.

Lastly, the \textbf{E6} variable queries the highest level of education completed by the respondent. We have encoded the responses to \textbf{E6} using the same threshold-based approach as the other variables. Responses less than 0 are assigned a value of -99, and values greater than or equal to 26 are set to 26, which corresponds to the highest level of education in the dataset. The remaining responses are classified into two ranges, $[0,9)$ and $[9,26)$, which represent the lower and higher levels of education, respectively.




\subsection{Experimental Settings}
\label{subsec:exp-settings}
In the previous section, we discussed the preprocessing steps taken to prepare the CTIS dataset for generating synthetic datasets. In this section, we will describe the experimental settings, including the detailed methodology and workflow explanation. Our aim is to provide different synthetic datasets using parametric and non-parametric data synthesizers described in section \ref{subsubsec:para} and \ref{subsubsec:non-para}. To achieve this, we carefully designed our methodology and selected specific variables to apply the synthesizing algorithms with a pre-defined order, based on their relevance and potential impact on COVID-19 transmission. Besides this, we will also explain each step of our workflow in a detailed illustration. Note that our empirical data synthesis experiments are based on the use of the \textit{synthpop} R package, developed by \citet{nowok2016synthpop}, in order to generate different synthetic datasets utilizing sequential modelling based synthesizers.

\subsubsection{Detailed Design of Methodology}
\label{subsubsec:design}
To describe the design the empirical experiment, we start by introducing the corresponding data types. Except for variable \textbf{weight}, the other variables are all categorical with at least 3 levels of inputs (given the presence of missingness -99 exists in each variable, except \textbf{weight}), where the input entries indicating various responses or an interval of response range. Considering the requirement to build up a more generic and convenient synthesizing mechanism, we simply encode all the other variables except the \textbf{weight} variable as integers. For instance, with variable \textbf{B2}, there are four different type of occurrences in the original dataset, including -1, -99, (0, 1], and [1, 3]. Note that given the original dataset dated from August 2nd to August 8th, 2020, there are no explicit explanations for input entries with value -1. However, we assume it is another pattern to record the missing data. To restore the original characteristics of the original dataset, we decide to keep the original inputs with value of -1 despite its comparably small number of entries. Moving on to the detailed encoding scheme for \textbf{B2}, we encode entries with -1 as 1, -99 with 2, [0,1) as 3, and [1,3) as 4. For variable \textbf{B4}, responses with value -99 are encoded as 1. Likewise, answers valued with [0, 1) and [1, 5) are encoded as 2 and 3, respectively. The next open response variable E5, for which the responses are thresholded with -99, [0, 1), and [1, 2), is encoded as integer 1, 2, and 3, accordingly. Last but not least, as for variable \textbf{E6}, the corresponding encoding scheme is formatted as: answers belonging to -99 are encoded as 1, and answers valued with [0, 9) are encoded with 2. To put in a more generic and general sense, given the number of occurrences existed in a certain variable, the correct encoding scheme is to assign the same number of integers to each occurrence group in an ascending order, where those open response variables discussed before are excluded. Note that more details related to the encoding scheme provided for each variable can be found in appendix \ref{subsec:encodescheme}.

When it comes to the design of methodology for data synthesis, given we have introduced the parametric and non-parametric synthesizers in chapter \ref{subsubsec:para} and \ref{subsubsec:non-para}, the next step is to consider the reasonable combination of data synthesizers applied on each variable. Since we have decided to use the powerful \textit{synthpop} R package to provide different synthetic datasets, a table with detail built-in synthesizing methods belonging to the non-parametric group, parametric group, and other group, is shown by Table \ref{tab:syn}.

\begin{table}[h]
    \centering
    \caption{Synthesizing methods to use in the experiment.}
    \begin{NiceTabular}{@{}lll@{}}[colortbl-like]\hline 
        \rowcolor{white!90} Method & Description & Data type \\\hline
        \rowcolor{white!90} \textit{Non-parametric} & & \\
        \rowcolor{white!90} \textbf{cart} & Synthesis with CART & Any\\
        \rowcolor{white!90} \textbf{rf} & Synthesis with random forest & Any\\
        \rowcolor{white!90} \textbf{bag} & Synthesis with bagging & Any\\
        \rowcolor{white!90} \textit{Parametric} & & \\
        \rowcolor{white!90} \textbf{norm} & Synthesis by normal linear regression & Numeric\\
        \rowcolor{white!90} \textbf{normrank} & Synthesis by normal linear regression preserving & Numeric\\
        \rowcolor{white!90}  & the marginal distribution & \\
        \rowcolor{white!90} \textbf{polyreg} & Synthesis by unordered polytomous regression & Factor, $>2$ levels\\
        \rowcolor{white!90} \textit{Other} & & \\
        \rowcolor{white!90} \textbf{sample} & Synthesis by random sampling & Any\\\hline
        \rowcolor{white!90} & & \\
        
        
        % \rowcolor{lightblue!20} \textbf{Section B} & \begin{tabular}[c]{@{}c@{}}B1\_1, B1\_2, until B1\_13, \\ B2, B3, until B11,  \\ B12\_1, B12\_2 until B12\_6\end{tabular}  \\\hline
        % \rowcolor{lightblue!60} \textbf{Section C} & \begin{tabular}[c]{@{}c@{}}C1\_m, C2, C3,\\ C5, C6, C7, C8\end{tabular}                                                     \\\hline
        % \\[-0.8em]
        % \rowcolor{lightblue!20} \textbf{Section D} & D1, D2, until D9                                                                  \\\hline
        % \rowcolor{lightblue!60} \textbf{Section E} & \begin{tabular}[c]{@{}c@{}}E2, E3, E4,\\ E5, E6, E7\end{tabular}                                                             \\\hline
        % \\[-0.8em]
        % \rowcolor{lightblue!20} \textbf{Section F} & F1, F2\_1, F2\_2                                                                  \\\hline
        % % \\[-0.8em]
        % \rowcolor{lightblue!60} \textbf{Other variables} & weight  %\\\hline
    \end{NiceTabular}
    {\parbox{6in}{
    \footnotesize Note that \textbf{cart} denotes the CART data synthesizer, \textbf{rf} indicates the random forest data synthesizer, \textbf{bag} represents the bagging based data synthesis algorithm, \textbf{polyreg} implies the polytomous logistic regression synthesizing method, \textbf{norm} stands for the normal linear regression data synthesizer, and \textbf{normrank} is an improved version of  norm by preserving the marginal distribution. By default, the \textit{synthpop} package uses the \textbf{cart} synthesizing method.}
    }
    % \vspace{1ex}
    % {\raggedright \par}

    \label{tab:syn}

\end{table}

In regard to the detailed design of the data synthesis methodology, we utilized the \textit{syn()} function built in the \textit{synthpop} R package, which offers a convenient tool for generating synthetic datasets. Based on Table \ref{tab:syn}, which presents different data synthesis algorithms accompanied by the corresponding compatible data types, we categorized our 54 variables into two groups: the normal variable group and the weight group. The normal group consisted of variables of interest that are closely related to the spread and transmission of COVID-19, including contacts-related variables, demographics-related variables, and app-related variables such as \textbf{B2}, \textbf{C1\_m}, and \textbf{E6}, etc. For the \textbf{weight} group, we simply examined the \textbf{weight} variable, which indicated the survey weighting adjusted to the Facebook user population.

Given the two types of variable groups, the generation of synthetic datasets involved a two-step procedure. The first step involved data synthesis with the normal variables of interest, and the second step involved synthesizing with the \textbf{weight} numeric variable. Moreover, as we implemented an encoding scheme for every variable except \textbf{weight}, we could apply both parametric and non-parametric data synthesizers, such as \textbf{norm}, \textbf{normrank}, \textbf{cart}, \textbf{rf}, \textbf{bag}, and \textbf{polyreg}, in the first step. For the second procedure, we chose from sample, \textbf{norm}, and \textbf{normrank} to implement synthesizing algorithms for the numeric weight variable. Thus, in total, 18 synthetic datasets were generated in the experiment.

In terms of implementation details in the R package \textit{synthpop}, the \textit{syn()} functionality has two essential parameters: the \textbf{method} parameter and the \textbf{visit.sequence} parameter. The \textbf{method} parameter indicated the record of data synthesizers applied to each variable maintained in the original dataset. This was formatted as a list of strings, with every entry of synthesis method corresponding to its variable, maintaining the same order as the original dataset. Note that variables that did not require synthesis had an empty method "". By default, all variables were synthesized using the \textbf{cart} data synthesizer. The \textbf{visit.sequence} parameter representes a character vector of names of variables or an integer vector of their column indices, which also specified the order of data synthesis. The default sequence of 1 to the total number of variables implied that column variables are synthesized from left to right. However, in our experiment, the \textbf{weight} parameter represented the survey weighting assigned to each data instance given the whole population. Therefore, we synthesized the second column (\textbf{B1\_1}) to the last column (\textbf{E6}) first and synthesized the \textbf{weight} variable last. It is worth noting that variable \textbf{E3} queried on the gender question, where the random sampling synthesizing method was sufficient for synthesizing.

\subsubsection{Workflow Explanation}
\label{subsubsec:workflow}

\begin{figure}[H]
    \centering
    \includegraphics[width=1\linewidth]{graphics/Fig-3-workflow.png}    
    \caption{Detailed workflow of the experiment design.}
    \label{fig:workflow}
    \floatfoot{Note: \textbf{ods} denotes the original dataset while \textbf{sds} represents the synthetic dataset.}
\end{figure}
Before delving into the details of our experiment, let us first take a look at Figure \ref{fig:workflow}, which presents the step-by-step workflow of the data synthesis process. The workflow can be divided into four main stages: data preprocessing, data synthesis stage 1, data synthesis stage 2, and data evaluation. The first stage involves the preprocessing of the original dataset, including the filtering of only european countries, the removal of unnecessary variables, and the encoding scheme for specific variables. The second and third procedures involve the selection of appropriate data synthesizers from a range of parametric and non-parametric methods, such as \textbf{norm}, \textbf{normrank}, \textbf{cart}, \textbf{rf}, \textbf{bag}, and \textbf{polyreg}, to be applied on the normal variables of interest. Notably, the \textbf{E3} variable is synthesized using random sampling, given our prior knowledge. The final step is to choose from \textbf{norm} and \textbf{normrank} to synthesize for the survey weighting variable. 

By following this step-by-step approach, we are able to generate a total of 18 synthetic datasets, allowing us to perform a thorough comparison of the performance of the different data synthesizers. The remaining evaluation steps include assessing the data utility, risk disclosure, and statistical inferences from fitted linear regression models. Organizing the experiment in this manner provides a logical and systematic approach, facilitating the comparison of the parametric and non-parametric data synthesizers. Moreover, to gain a broader perspective on the performance of statistics-based and generator-based data synthesizing methods, we also evaluate the synthetic datasets produced by \citet{liu2021iterative}, which makes a total of 20 synthetic datasets for further evaluation purposes to conduct a comparison between different data synthesizers' performances.

Overall, this step-by-step workflow shown in Figure \ref{fig:workflow} provides a clear and comprehensive framework for conducting our experiment, allowing us to effectively assess the quality and utility of the generated synthetic datasets.




\subsection{Results Generated from Exploratory Analysis}
\label{subsec:results}
After generating the 20 synthetic datasets, the subsequent phase of our experiment is to assess their data utility, perform a risk disclosure analysis, and draw statistical inferences through the use of fitted linear regression models to compare the corresponding fitting with regard to each parameter.

\subsubsection{Data Utility}
\label{subsubsec:datautility}
Moving on to the evaluation of data utility, a key aspect of our experiment is the use of global utility metrics for propensity score matching. This is an appropriate approach, given the context of our experiment design and prior knowledge provided. To this end, we employ the standardized $p_{MSE}$, denoted as $Sp_{MSE}$, to evaluate the utility of the synthetic datasets. Specifically, we calculate the mean squared error (MSE) of the predictions relative to the true values, and normalize it by a measure of the variability in the true values. This approach allows for comparison of models on different datasets with different means and standard deviations. A lower $Sp_{MSE}$ value indicates a higher analytical validity of the synthetic data, and a synthesized column with $Sp_{MSE}$ scoring less than 10 can be regarded as a good fit in regard to a single variable synthesizing performance. The standardized $p_{MSE}$, denoted as $Sp_{MSE}$, is utilized in our study to evaluate data utility. As described in Section \ref{subsubsec:global}, $p_i$ and $t_i$ represent the $i$th prediction and the corresponding true value, respectively. The standardized $p_{MSE}$ is computed as:
\begin{align}
\label{eqn:spmse}
\begin{split}
    \frac{\frac{1}{n} \sum_{i=1}^n (p_i - t_i)^2}{\left(\frac{1}{n} \sum_{i=1}^n t_i\right)^2 + \text{Var}(t)},
\end{split}
\end{align}
where $n$ represents the total number of predictions and true values, and $\text{Var}(t)$ is the variance of the true values. The equation \eqref{eqn:spmse} indicates that we are calculating the mean squared error (MSE) of the predictions relative to the true values, and normalizing it by a measure of the variability in the true values. The denominator is the sum of the squared mean and the variance of the true values, which allows for comparison of models on different datasets with different means and standard deviations. A lower $Sp_{MSE}$ value indicates higher analytical validity of the synthetic data. A synthesized column with $Sp_{MSE}$ scoring less than 10 can be regarded as having a comparably good fit in terms of single-variable synthesizing performance.

To evaluate data utility, we examine all the $Sp_{MSE}$ scores with regard to each variable in the corresponding synthetic dataset. Using the \textit{compare()} function in the \textit{synthpop} package, we generate Table \ref{tab:spmse}, which shows the number of synthesizing variables in which $Sp_{MSE}<10$.

\begin{table}[h]
    \centering
    \caption{The number of synthesis variables with a score of $Sp_{MSE}<10$.}
    \begin{NiceTabular}{@{}ll@{}}[colortbl-like]\hline 
        \rowcolor{white!90} Synthsis combinations & Number\\\hline
        \rowcolor{white!90} \textit{cart} & \\
        \rowcolor{white!90} sds\_cart\_sample & 45\\
        \rowcolor{white!90} sds\_cart\_norm & 43\\
        \rowcolor{white!90} sds\_cart\_normrank & 43\\\hline
        \rowcolor{white!90} \textit{rf} & \\
        \rowcolor{white!90} sds\_rf\_sample & 28\\
        \rowcolor{white!90} sds\_rf\_norm & 30\\
        \rowcolor{white!90} sds\_rf\_normrank & 27\\\hline
        \rowcolor{white!90} \textit{bag} & \\
        \rowcolor{white!90} sds\_bag\_sample & 37\\
        \rowcolor{white!90} sds\_bag\_norm & 38\\
        \rowcolor{white!90} sds\_bag\_normrank & 37\\\hline
        \rowcolor{white!90} \textit{polyreg} & \\
        \rowcolor{white!90} sds\_polyreg\_sample & 43\\
        \rowcolor{white!90} sds\_polyreg\_norm & 42\\
        \rowcolor{white!90} sds\_polyreg\_normrank & 42\\\hline
        \rowcolor{white!90} \textit{norm} & \\
        \rowcolor{white!90} sds\_norm\_sample & 28\\
        \rowcolor{white!90} sds\_norm\_norm & 25\\
        \rowcolor{white!90} sds\_norm\_normrank & 26\\\hline
        \rowcolor{white!90} \textit{normrank} & \\
        \rowcolor{white!90} sds\_normrank\_sample & 27\\
        \rowcolor{white!90} sds\_normrank\_norm & 27\\
        \rowcolor{white!90} sds\_normrank\_normrank & 26\\\hline
    \end{NiceTabular}
    {\parbox{6in}{
    \footnotesize Note that column "number" specifies the number of variables which suffice the $Sp_{MSE}<10$ requirement.}
    }
    \label{tab:spmse}

\end{table}

In terms of evaluation of synthesizing performance based on the scoring of $Sp_{MSE}$, it is pretty clear that generally those non-parametric data synthesizers outperform those parametric ones, within which the \textbf{cart} synthesizing method has shown the best performances regardless of which weight synthesizing algorithm to use during synthesis stage 2, scoring with at least 43 variables indicating a higher analytical validity in the synthesized datasets. At the same time, the \textbf{polyreg} data synthesizer slightly fell behind \textbf{cart} by at most a gap of three variables. In contrast, parametric synthesizers based on linear regression own a average of 26 variables which suffice the $Sp_{MSE}<10$ requirement. 

XXXX more to add regarding the which variables generally perform well + model complexity

compare, utility.table(), S mMSE

With regards to data utility, there exists one variable that performs notably inferior to the others, where the standardized propensity score measure shows that it deviates by at least 2,000. Such a significant deviation suggests that the synthesized records from this variable are likely to be unreliable, leading to potential challenges in conducting valid statistical inferences. Thus, additional efforts may need to be taken to address the challenges brought by this variable to ensure the overall quality of the synthesized dataset.

The comparison between non-parametric and parametric data synthesizers reveals that the former outperforms the latter in terms of data utility. This finding indicates that non-parametric synthesizers are more effective in capturing the complex relationships between variables, resulting in synthesized datasets that better approximate the true data distribution. As such, researchers who seek to generate high-quality synthetic datasets may benefit from utilizing non-parametric data synthesizers over their parametric counterparts.

Although non-parametric data synthesizers are known to produce more accurate results, this is not always the case, especially when models with higher complexity are used. In some cases, higher model complexity may lead to overfitting, which can result in synthesized records that have low validity with regard to data utility. Thus, researchers who wish to utilize non-parametric data synthesizers may need to carefully consider the complexity of the models they use to ensure that the synthesized records are accurate and reliable.

In the context of parametric synthesizing, the \textbf{normrank} synthesizer outperforms the \textbf{norm} synthesizer in terms of data utility. This finding can be attributed to the fact that the \textbf{normrank} synthesizer maintains the marginal distribution of the original dataset, resulting in a more aligned distribution in the synthesized dataset. As such, researchers who seek to generate synthetic datasets with high data utility may benefit from utilizing the \textbf{normrank} synthesizer over the \textbf{norm} synthesizer in their parametric data synthesizing applications.

\subsubsection{Replicated Uniques and New Row Synthesis}
\label{subsubsec:sde}

\subsubsection{Inference from Fitted Linear Regression Models}
\label{subsubsec:inferencelm}
estimate, CI-overlap, ggplot

computation issues also can be included with rf, bag.

for vars, for non-para vs. para
\subsection{Other Findings}
\label{subsec:findings}
, for model complexity, , for synthesis with missingness
\newpage

% \section{General Discussion}
% \label{chapter6:discussion}
% \input{chapters/ch6_discussion}

\section{Conclusion}
\label{chapter6:conclusion}
In this thesis, we evaluated different synthetic datasets generated using parametric, non-parametric, and generator network-based data synthsis algorithms in the study to produce synthetic data for the Covid-19 Trend and Impact (CTIS) dataset. We reviewed the state of technology proposed for synthetic data generation and selected a few algorithms for extensive application. Subsequently, we proposed a experiment framework with 4 steps (step 1: data preprocessing; Step 2: data synthesis for categorical variables; Step 3: data synthesis for numerical variables; Step 4: utility evaluation and risk assessment;) and especially to deal with categorical and numerical variables and examine the difference between the performance of data synthesizers applied on them, the data synthesis phrase was broken down into two stages, respectively. Finally, we experimented with selected data synthesizers applied on the original data scource to determine the quality of various synthetic datasets in terms of overall data utility, univariate data utility, analysis-specific utility (by fitting linear regression model with selected predictors), and replication analysis.

We conclude our work by re-examining the research objective and assess which synthetic data generation actually has the potential to be used in a real-world environment by comparing their corresponding data utility and risk assessment. We evaluate the strenghts and limitations of different algorithms based data synthesizers and discuss its relevance and utility. Finally we provide insights into future research opportunities.

\subsection{Review of Research Objective}
\label{subsec:review}
Our experiments revealed that non-parametric data synthesizers outperformed parametric data synthesizers in terms of general propensity score measure and analysis-specific utility evaluation. Specifically, decision tree and polytomous regression-based methods demonstrated the best performance, particularly with respect to the standardized propensity score measure. Moreover, the decision tree-based data synthesizer ranked the highest in terms of analysis-specific utility evaluation, as evidenced by fitting linear regression models with synthetic values compared to the original dataset. Nevertheless, our study also highlighted the potential limitation of higher model complexity in constructing non-parametric data synthesizers, which may result in the failure to restore the corresponding data utility.

Regarding univariate data utility, we observed that the synthesizing of categorical variables generally performed well when no newly generated records were involved. However, synthesizing numerical variables could be more challenging, resulting in a loss of data utility compared to synthesizing binary variables with fewer levels. This is mainly due to the difficulty in preserving complex relationships and patterns between different variables in the original dataset. Additionally, data synthesizers demonstrated poor performance in synthesizing categorical variables with a substantial number of levels.

In addition to evaluating data utility, we also performed a risk disclosure analysis using a replicated uniques score. Our study demonstrated that parametric data synthesizers based on linear regression outperformed other methods due to the generation of new records that do not exist in the original dataset.

Notably, the GEM-based data synthesizer, a generator network-based algorithm, generally demonstrated the worst performance in terms of data utility evaluation. This finding may be due to the inability of neural network-based data synthesizing algorithms to capture the interactions between query data. Unlike image or audio data, which have spatial interactions between cells, query data is typically stored as tabular data, making it challenging for these methods to detect relationships between variables.

\subsection{Limitation and Future work}
\label{subsec:future}
One potential limitation of our study is the lack of tuning of inbuilt parameters in each data synthesizer during the categorical data synthesis process. For instance, we did not vary the number of trees grown by the random forest and bagging-based data synthesizers, which were kept constant at 10. By exploring different parameter settings, it may be possible to generate synthetic data with even higher utility, which is an avenue for future research.

Looking ahead, we could investigate the use of other data synthesizers and compare their performance with the methods employed in this study. For instance, we could explore the application of generative adversarial networks (GANs) variational autoencoders (VAEs), and differential privacy (DP) for synthetic data generation. Moreover, we could examine the performance of data synthesizers on diverse datasets with varying sizes and structures, which may help identify the most effective methods for different data types. Finally, we could explore the potential benefits of combining multiple data synthesizers to generate high-quality synthetic datasets. These research directions hold the promise of advancing the field of synthetic data generation and enhancing the use of synthetic data in diverse domains.

\newpage

% ------------------------------------------------------------------------------
% APPENDIX ---------------------------------------------------------------------
% ------------------------------------------------------------------------------
    
\pagenumbering{Roman}

\setcounter{page}{5} % CHANGE

\appendix

\section{Appendix}
\label{app}
Additional material goes here

\subsection{A Detailed List of Variable Meanings}
\label{subsec:listofvars}

\subsection{The GEM Algorithm}
\label{subsec:gem}

\subsection{The Encoding Scheme Prepared for Variables}
\label{subsec:encodescheme}
\newpage

\section{Electronic appendix}
\label{el_app}

Data, code and figures are provided in electronic form.

\newpage
    
% ------------------------------------------------------------------------------
% BIBLIOGRAPHY -----------------------------------------------------------------
% ------------------------------------------------------------------------------

\RaggedRight
\bibliography{ref}
\bibliographystyle{dcu}
\newpage

% ------------------------------------------------------------------------------
% DECLARATION OF AUTHORSHIP-----------------------------------------------------
% ------------------------------------------------------------------------------

\Large
\noindent
\textbf{Declaration of authorship} 
\vspace{0.5cm}
\noindent
\normalsize

I hereby declare that the report submitted is my own unaided work. All direct 
or indirect sources used are acknowledged as references. I am aware that the 
Thesis in digital form can be examined for the use of unauthorized aid and in 
order to determine whether the report as a whole or parts incorporated in it may 
be deemed as plagiarism. For the comparison of my work with existing sources I 
agree that it shall be entered in a database where it shall also remain after 
examination, to enable comparison with future Theses submitted. Further rights 
of reproduction and usage, however, are not granted here. This paper was not 
previously presented to another examination board and has not been published.
\\

\vspace{1cm}
\textcolor{orange}{Location, date} \\

\vspace{3cm}

\noindent\rule{0.5\textwidth}{0.4pt} \\

\textcolor{orange}{Name}

% ------------------------------------------------------------------------------

\end{document}
